\documentclass[12pt]{article}
% add [leqno] to move equation numbers to left side

%%%%%%%%%%%%%%%%%%%%%%%%%%%%%%%%%%
%%%%%%%%%%%%    Packages    %%%%%%%%%%%%%%%
%%%%%%%%%%%%%%%%%%%%%%%%%%%%%%%%%%

%%%%%%%%%%%%%%%%%%%%%%%%%%%%%%%%%%
%%%%%%%%%%   Formatting Packages    %%%%%%%%%%%
%%%%%%%%%%%%%%%%%%%%%%%%%%%%%%%%%%
%\usepackage{showframe} % draw boxes around body, etc. to detect broken margins
%\pdfpkresolution=1200
\usepackage{fancyhdr}
\usepackage{enumitem} % Better Lists
\newlist{condenum}{enumerate}{1}
%\usepackage{nicematrix} % Better matrices
\usepackage{xcolor} % Better colors
\usepackage{graphicx} % Better layout control
\usepackage{hyperref} % Better \refs and \cites

%%%%%%%%%%%%%%%%%%%%%%%%%%%%%%%%%%
%%%%%%%%%%%    Math Packages    %%%%%%%%%%%%%
%%%%%%%%%%%%%%%%%%%%%%%%%%%%%%%%%%
\usepackage{amsmath}
\allowdisplaybreaks
\usepackage{amsthm}
\usepackage{amssymb}
\usepackage{mathtools}
\usepackage{mathabx}
\usepackage{bbm}
\usepackage{shuffle}
\usepackage[mathscr]{eucal}
%\usepackage{mathrsfs}
\usepackage{stmaryrd} % Includes \mapsfrom
\usepackage{makecell}

%%%%%%%%%%%%%%%%%%%%%%%%%%%%%%%%%%
%%%%%%%%%%%    TikZ Packages    %%%%%%%%%%%%%
%%%%%%%%%%%%%%%%%%%%%%%%%%%%%%%%%%
\usepackage{tikz}
\usetikzlibrary{patterns, arrows, matrix, positioning, calc}
\usepackage{tikz-cd}


%%%%%%%%%%%%%%%%%%%%%%
% fancy comments
%%%%%%%%%%%%%%%%%%%%%%
\usepackage{xcolor}
\usepackage[colorinlistoftodos]{todonotes}
\newcommand{\lucas}[1]{\todo[size=\tiny,color=violet!30]{#1 \\ \hfill --- Sami}}
\newcommand{\Lucas}[1]{\todo[size=\tiny,inline,color=violet!30]{#1 \\ \hfill --- Sami}}
\newcommand{\nantel}[1]{\todo[size=\tiny,color=blue!30]{#1 \\ \hfill --- Nantel}}
\newcommand{\Nantel}[1]{\todo[size=\tiny,inline,color=blue!30]{#1 \\ \hfill --- Nantel}}


%%%%%%%%%%%%%%%%%%%%%%%%%%%%%%%%%%
%%%%%%%%%%%%    Page setup    %%%%%%%%%%%%%%
%%%%%%%%%%%%%%%%%%%%%%%%%%%%%%%%%%
\setlength{\evensidemargin}{1in}
\addtolength{\evensidemargin}{-1in}
\setlength{\oddsidemargin}{1in}
\addtolength{\oddsidemargin}{-1in}
\setlength{\topmargin}{1in}
\addtolength{\topmargin}{-1.5in}
%
\setlength{\textwidth}{16.5cm}
\setlength{\textheight}{23cm}




%%%%%%%%%%%%%%%%%%%%%%%%%%%%%%%%%%
%%%%%%%%%%%     Bibliography     %%%%%%%%%%%%%%
%%%%%%%%%%%%%%%%%%%%%%%%%%%%%%%%%%
%\usepackage[
%	backend = biber,
%	style=numeric,
%	natbib=false,
%	isbn=false, 
%	url = false,
%	isbn=false, 
%	doi=false]{biblatex}
%\addbibresource{}

%%%%%%%%%%%%%%%%%%%%%%%%%%%%%%%%%%
%%%%%%%%%%     Macros - General     %%%%%%%%%%%%
%%%%%%%%%%%%%%%%%%%%%%%%%%%%%%%%%%

%%%%%%%%%%%
% Theorem Environments
%%%%%%%%%%%
\newtheorem{thm}[equation]{Theorem}
\newtheorem{conj}[equation]{Conjecture}
\newtheorem{openproblem}[equation]{Open Problem}
\newtheorem{question}[equation]{Question}
\newtheorem{prop}[equation]{Proposition}
\newtheorem{lem}[equation]{Lemma}
\newtheorem{cor}[equation]{Corollary}
\theoremstyle{definition}
\newtheorem{ex}[equation]{Example}
\newtheorem{definition}[equation]{Definition}
\theoremstyle{remark}
\newtheorem{rem}[equation]{Remark}
\newtheorem{rems}[equation]{Remarks}
%
\numberwithin{equation}{section}
%

%%%%%%%%%%%
% Letter Macros
%%%%%%%%%%%
% Blackboard Letters
\newcommand{\FF}{\mathbb{F}}
\newcommand{\CC}{\mathbb{C}}
\newcommand{\RR}{\mathbb{R}}
\newcommand{\QQ}{\mathbb{Q}}
\newcommand{\ZZ}{\mathbb{Z}}
\newcommand{\NN}{\mathbb{N}}
\newcommand{\KK}{\mathbb{K}}
% Calligraphic - Triple Capital Letters
\def\mydefb#1{\expandafter\def\csname #1#1#1\endcsname{\mathcal{#1}}}
\def\mydefallb#1{\ifx#1\mydefallb\else\mydefb#1\expandafter\mydefallb\fi}
\mydefallb ABCDEFGHIJKLMNOPQRSTUVWXYZ\mydefallb

%%%%%%%%%%%
% Nicening Macros
%%%%%%%%%%%
\renewcommand{\epsilon}{\varepsilon}
\newcommand{\spanning}{\operatorname{-span}}
\renewcommand{\hom}{\operatorname{Hom}}
\renewcommand{\setminus}{-}

%%%%%%%%%%%
% Width Adjustment
%%%%%%%%%%%
% For sufficiently new LaTeX installations:
%\NewCommandCopy{\narrowhat}{\hat}
%\renewcommand{\hat}{\widehat}
%\NewCommandCopy{\shortto}{\to}
%\renewcommand{\to}{\longrightarrow}
%\NewCommandCopy{\shortmapsto}{\mapsto}
%\renewcommand{\mapsto}{\longmapsto}
%\NewCommandCopy{\shortmapsfrom}{\mapsfrom}
%\renewcommand{\mapsfrom}{\longmapsfrom}

%%%%%%%%%%%%%%%%%%%%%%%%%%%%%%%%%%
%%%%%%%%%%%     Macros - Paper     %%%%%%%%%%%%
%%%%%%%%%%%%%%%%%%%%%%%%%%%%%%%%%%

\newcommand{\TL}{\mathsf{TL}}

\newcommand{\Sym}{\mathrm{Sym}}
\newcommand{\QSym}{\mathrm{QSym}}
\newcommand{\QSV}{\mathrm{QSV}}
\newcommand{\NCP}{\mathrm{NCP}}

\newcommand{\EP}{E_{pos}}
\newcommand{\EV}{E_{val}}

%%%%%Tikz Paper Macros
%% Black edge
\newcommand{\edge}[2]{\tikz[scale = 0.75, baseline = -0.2cm]{
\draw (0.1, 0) node[inner sep = 0ex] (a) {};
\node[left, xshift = 0.1cm, yshift = -0.1cm] at (a) {$\scriptstyle #1$};
\draw (0.9, 0)  node[inner sep = 0ex] (b) {};
\node[right, xshift = -0.1cm, yshift = -0.1cm] at (b) {$\scriptstyle #2$};
\draw[thick] (a) to[out = 35, in = 145] (b);}}

%%%%%%%%%%%%%%%%%%%%%%%%%%%%%%%%%%
%%%%%%%%%%%     Page Details     %%%%%%%%%%%%%%
%%%%%%%%%%%%%%%%%%%%%%%%%%%%%%%%%%
\pagestyle{fancy}
%\fancyhf{}
\lhead{}
\chead{}
\rhead{}
\cfoot{\thepage}
%%%%%%%%%%%%%%%%%%%%%%%%%%%%%%%%%%

\begin{document}

\title{The excedance quotient of the Bruhat order, Quasisymmetric Varieties and Temperley-Lieb algebras}
\author{Lucas and Nantel}
%\date{\day}
\maketitle
%\begin{abstract}
%\end{abstract}
%\keywords{Noncrossing partitions, Quasisymmetric polynomials, Varieties of points}

%%%%%%%%%%%%%%%%%%%%%%%%%%%%%%%%%%%%%%%%
\section{Introduction}
%%%%%%%%%%%%%%%%%%%%%%%%%%%%%%%%%%%%%%%%

%%%%%%%%%%%%%%%%%%%%%%%%%%%%%%%%%%%%%%%%
\section{Preliminaries}
%%%%%%%%%%%%%%%%%%%%%%%%%%%%%%%%%%%%%%%%

%%%%%%%%%%
\subsection{Noncrossing partitions}
\label{sec:ncp}
%%%%%%%%%%

%For $n \ge 0$, let $[n]$ denote the set comprising the first $n$ positive integers.

A \emph{noncrossing partition} of size $n$ is a diagram $\lambda$ consisting of:
\begin{enumerate}
\item the positive integers $1, \ldots, n$, placed from left to right along a horizontal axis; and

\item a set of left-to-right arcs $ (i, j) = \edge{i}{j}$, $i < j$ drawn above the axis with no intersections or coterminal points: $\lambda$ contains no pair $\edge{i}{k}, \edge{j}{l}$ with $i \le j < k \le l$.

\end{enumerate}
For example,
\begin{equation}
\label{eq:noncrossingpartitionexample}
\lambda = \begin{tikzpicture}[scale = 0.75, baseline = 0.75*-0.2]
\foreach \x in {1, ..., 7}{\draw[fill] (\x - 1, 0) node[inner sep = 2pt] (\x) {$\scriptstyle \x$};}
%\foreach \x in {1, ..., 7}{\node[below] at (\x) {$\scriptstyle \x$};}
\foreach \i\j in {5/6, 3/5,2/7}{\draw[thick] (\i) to[out = 35, in = 145] (\j);}
\end{tikzpicture}
\end{equation}
is a noncrossing partition of size $7$ containing the arcs $\edge{2}{7}$, $\edge{3}{5}$, and $\edge{5}{6}$.

Considering a noncrossing partition $\lambda$ as an (undirected) graph, the connected components of $\lambda$ give a partition of the set $[n] = \{1, \ldots, n\}$, which is the origin of the term.  For example, the noncrossing partition shown in Equation~\eqref{eq:noncrossingpartitionexample} corresponds to the set partition $\{ \{1\}, \{3, 5, 6\}, \{2, 7\}, \{4\}  \}$.  Let
\[
\NCP_{n} = \{ \text{noncrossing partitions of size $n$} \}.
\]
The number of noncrossing partitions of size $n$ is the $n$th Catalan number, $C_{n} = \frac{1}{n+1}\binom{2n}{n}$ (see~\cite{Stanley}).

Given an arc $ \edge{i}{j} \in \lambda$, say that $i$ is the \emph{left endpoint} and $j$ is the \emph{right endpoint}, and let
\[
\lambda^{+} = \{ i \in [n] \;|\; \text{$i$ is a left endpoint in $\lambda$} \}
\]
and
\[
\lambda^{-} = \{ i \in [n] \;|\; \text{$i$ is a right endpoint in $\lambda$} \}
\]
For example, with the noncrossing partition $\lambda$ in~\eqref{eq:noncrossingpartitionexample}, $\lambda^{+} = \{2, 3, 5\}$ and $\lambda^{-} = \{5, 6, 7\}$.  The arcs in $\lambda$ give a bijection between the sets $\lambda^{+}$ and $\lambda^{-}$, so that
\[
|\lambda^{+}| = |\lambda^{-}|.
\]

The following lemma is classic in the literature about non-crossing partitions (for example, see~\cite{Stanley}). We include a proof for the sake of exposition and the way we will use it in the following.
%Each noncrossing partition $\lambda$ is uniquely determined by its sets $\lambda^{+}$ and $\lambda^{-}$ of left and right neighbors.
\begin{lem}
\label{lem:noncrossingpartitionproperty}
Each noncrossing partition $\lambda$ of size $n$ is uniquely determined by the sets $\lambda^{+}$ and $\lambda^{-}$.  Moreover, given two subsets $L$ and $R$ of $[n]$ with equal size, the inequalities
\[
|[k-1] \cap L| \ge |[k] \cap R| \qquad\text{for $k \ge 1$}
\]
hold if and only if we have $L = \lambda^{+}$ and $R = \lambda^{-}$ for some noncrossing partition $\lambda$ of size $n$.
\end{lem}
\begin{proof}
Given $L$ and $R$, draw the elements of $[n]$ along the horizontal axis, increasing from left to right, and draw a half-arc starting at each vertex in $L$ and and half-arc ending at each vertex in $R$.  For example, with $n = 8$, $L = \{2, 4, 7\}$, and $R = \{5, 6, 8\}$, the resulting diagram is
\[
\begin{tikzpicture}[scale = 0.75, baseline = 0.75*-0.2]
\foreach \x in {1, ..., 8}{\draw[fill] (\x - 1, 0) circle (2pt) node[inner sep = 2pt] (\x) {};}
\foreach \x in {1, ..., 8}{\node[below] at (\x) {$\scriptstyle \x$};}
\foreach \i\j in {2/6, 4/5, 7/8}{
	\draw[thick, -latex] (\i) to[out = 35, in = 215] ($ (\i) + (0.4, 0.3) $);
	\draw[thick, -latex] ($ (\j) + (-0.4, 0.3) $) to[out = -35, in = 145] (\j);}
\end{tikzpicture}.
\]
These half-arcs determine a unique noncrossing partition, which can be obtained by recursively connecting pairs of half-edges which have no incomplete edges between them. 
This is the same process as matching open and closed parenthesis, the difference here is that there are displayed in prescribe positions. The condition on $L$ and $R$ guaranties 
that there is always a starting half-arc in $L$ available to match any ending half-arc in $R$. The non-crossing condition follows from the choosing process: a crossing would contradict a choice we 
made earlier.
Continuing the preceding example, this successively gives the  diagrams
\[
\begin{tikzpicture}[scale = 0.75, baseline = 0.75*-0.2]
\foreach \x in {1, ..., 8}{\draw[fill] (\x - 1, 0) circle (2pt) node[inner sep = 2pt] (\x) {};}
\foreach \x in {1, ..., 8}{\node[below] at (\x) {$\scriptstyle \x$};}
\foreach \i\j in {2/6}{
	\draw[thick, -latex] (\i) to[out = 35, in = 215] ($ (\i) + (0.4, 0.3) $);
	\draw[thick, -latex] ($ (\j) + (-0.4, 0.3) $) to[out = -35, in = 145] (\j);}
\foreach \i\j in {4/5, 7/8}{\draw[thick] (\i) to[out = 35, in = 145] (\j);}
\end{tikzpicture}
\qquad\text{and then}\qquad
\begin{tikzpicture}[scale = 0.75, baseline = 0.75*-0.2]
\foreach \x in {1, ..., 8}{\draw[fill] (\x - 1, 0) circle (2pt) node[inner sep = 2pt] (\x) {};}
\foreach \x in {1, ..., 8}{\node[below] at (\x) {$\scriptstyle \x$};}
\foreach \i\j in {2/6, 4/5, 7/8}{\draw[thick] (\i) to[out = 35, in = 145] (\j);}
\end{tikzpicture}.
\]

For the converse, if $L = \lambda^{+}$ and $R = \lambda^{-}$ for a nonnesting partition $\lambda$, the given inequalities must hold: the left endpoint of each arc must be in a position strictly less than the right endpoint.
\end{proof}


%%%%%%%%%%
\subsection{Permutations and the Bruhat order}
\label{sec:bruhat}
%%%%%%%%%%

Let $S_n$ denote the set of permutations of $[n]$ and for $w\in S_n$ we denote by $\ell(w)$ the length of $w$, that is the number of inversions $i<j$ where $w_i>w_j$.
The Bruhat ordering of $S_{n}$ is generated by the covering relation 
\[
v \lessdot w \quad\text{if $wv^{-1}$ is a reflection and $\ell(v) +1 = \ell(w)$}.
\]
This definition above is somewhat unwieldy, so the following \emph{tableau criterion} will be used.  For $w \in S_{n}$, let $\TTT(w)$ be the diagram whose $n-k$th row $\TTT_{k}(w)$ consists of the elements $w_{1}, w_{2}, \ldots, w_{k}$ appearing in increasing order.  For example,
\[
\TTT(52314) = 
\begin{array}{ccccc} 
1 & 2 & 3 & 4 & 5 \\ 
1 & 2 & 3 & 5 \\ 
2 & 3 & 5 \\ 
2 & 5 \\ 
5
\end{array}
\qquad\text{and}\qquad
\TTT(41235) = \begin{array}{ccccc} 
1 & 2 & 3 & 4 & 5 \\ 
1 & 2 & 3 & 4 \\ 
1 & 2 & 4 \\ 
1 & 4 \\ 
4
\end{array}.
\]

\begin{prop}[{\cite[Theorem 2.6.3]{BjornerBrenti}}]
\label{TableauCriterion}
For $v, w \in S_{n}$, we have $v \le w$  if and only if each entry of $\TTT(v)$ is less than or equal to to corresponding entry of $\TTT(w)$.
\end{prop}

For example, with $w = 52314$ and $v = 41235$ and $\TTT(v)$, $\TTT(w)$ shown above, the Tableau criterion shows that $v \le w$.

%%%%%%%%%%%%%%%%%%%%%%%%%%%%%%%%%%%%%%%%
\section{The set $\QSV_{n}$}
\label{sec:QSV}
%%%%%%%%%%%%%%%%%%%%%%%%%%%%%%%%%%%%%%%%

This section will define the set $\QSV_{n}\subseteq S_n$ and establish its elementary properties.  Section~\ref{sec:bruhatballot} recalls the description of the restriction of the Bruhat order to $\QSV_{n}$ given in~\cite{GobetWilliams}.


Let $\lambda$ be a noncrossing partition of size $n$ and recall the sets $\lambda^{+}$ and $\lambda^{-}$ from Section~\ref{sec:ncp}.  
%, and write each part of $\lambda$ as 
%\[
%\lambda_{r} = \{c_{r, 1} < c_{r, 2} < \dots < c_{r, |\lambda_{r}|}\}.
%\]
Define a permutation $Q_{\lambda} \in S_{n}$ by 
%the product of disjoint cycles:
%\[
%Q_{\lambda} = \prod_{\lambda_{r} \in \lambda} (c_{r, 1} c_{r, 2} \dots c_{r, |\lambda_{r}|}).
%\]
%Since the parts of $\lambda$ are disjoint, the above product can be taken in any order.  The permutation $Q_{\lambda}$ can also be described using the noncrossing partition diagram of $\lambda$: 
\[
Q_{\lambda}(j) = \begin{cases} i & \text{if $j \in \lambda^{-}$ and $ \edge{i}{j} \in \lambda$} \\
k & \text{if $j \notin \lambda^{-}$ and $k$ is the largest element connected to $i$ in $\lambda$}
\end{cases}
\]
Thus, $Q_{\lambda}$ sends $j \in [n]$ to its leftward neighbor in $\lambda$, if such a neighbor exists, and otherwise sends $j$ to the rightmost element of its connected component.  

Let
\[
\QSV_{n} = \{Q_{\lambda} \;|\; \lambda \in \NCP_{n} \}.
\]
For example, the elements of $\QSV_{3}$ are:
\[
Q_{\begin{tikzpicture}[scale = 0.35, baseline = 0.35*-0.2]
\foreach \x in {1, ..., 3}{\draw[fill] (\x - 1, 0) circle (2pt) node[inner sep = 2pt] (\x) {};}
\foreach \x in {1, ..., 3}{\node[below] at (\x) {$\scriptstyle \x$};}
\foreach \i\j in {1/3}{\draw[thick] (\i) to[out = 35, in = 145] (\j);}
\end{tikzpicture}} = 321, \qquad
Q_{\begin{tikzpicture}[scale = 0.35, baseline = 0.35*-0.2]
\foreach \x in {1, ..., 3}{\draw[fill] (\x - 1, 0) circle (2pt) node[inner sep = 2pt] (\x) {};}
\foreach \x in {1, ..., 3}{\node[below] at (\x) {$\scriptstyle \x$};}
\foreach \i\j in {1/2, 2/3}{\draw[thick] (\i) to[out = 35, in = 145] (\j);}
\end{tikzpicture}} = \{312\}, \qquad
Q_{\begin{tikzpicture}[scale = 0.35, baseline = 0.35*-0.2]
\foreach \x in {1, ..., 3}{\draw[fill] (\x - 1, 0) circle (2pt) node[inner sep = 2pt] (\x) {};}
\foreach \x in {1, ..., 3}{\node[below] at (\x) {$\scriptstyle \x$};}
\foreach \i\j in {1/2}{\draw[thick] (\i) to[out = 35, in = 145] (\j);}
\end{tikzpicture}} = 213, 
\]
\[
Q_{\begin{tikzpicture}[scale = 0.35, baseline = 0.35*-0.2]
\foreach \x in {1, ..., 3}{\draw[fill] (\x - 1, 0) circle (2pt) node[inner sep = 2pt] (\x) {};}
\foreach \x in {1, ..., 3}{\node[below] at (\x) {$\scriptstyle \x$};}
\foreach \i\j in {2/3}{\draw[thick] (\i) to[out = 35, in = 145] (\j);}
\end{tikzpicture}} = 132, \qquad\text{and}\qquad
Q_{\begin{tikzpicture}[scale = 0.35, baseline = 0.35*-0.2]
\foreach \x in {1, ..., 3}{\draw[fill] (\x - 1, 0) circle (2pt) node[inner sep = 2pt] (\x) {};}
\foreach \x in {1, ..., 3}{\node[below] at (\x) {$\scriptstyle \x$};}
%\foreach \i\j in {2/3}{\draw[thick] (\i) to[out = 35, in = 145] (\j);}
\end{tikzpicture}} = 123.
\]
%The following well-known characterization of $\QSV_{n}$ will be used in Section~\ref{}.%; a proof is included for compelteness.

\begin{lem}
\label{lem:QSVcycles}

Let $\lambda$ be a noncrossing partition of size $n$ with connected components $C_{1}, C_{2}, \ldots, C_{s}$, and enumerate each $C_{r}$, $1 \le r \le s$ in increasing order as $\{c_{r, 1} < c_{r, 2} < \cdots < c_{r, |C_{r}|}\}$. 
We then have the disjoint cycle decomposition
\[
Q_{\lambda} = \prod_{r = 1}^{s} (c_{r, 1} c_{r, |C_{r}|} \cdots c_{r, 2}).
\]
\end{lem}
\begin{proof}
The statement follows directly from the definition given above: $[n] \setminus \lambda^{-} = \{c_1 \;|\; 1 \le r \le s\}$, so $Q_{\lambda}(c_{r, 1}) =  c_{r, |C_{r}|}$ for each $r$, and for $1 < i \le |C_{r}|$, we have $c_{r, i} = c_{r, i-1}$.
\end{proof}

For example, when $\lambda$ has a single connected component, $Q_{\lambda}$ is a single cycle: with
\[
\lambda = \begin{tikzpicture}[scale = 0.75, baseline = 0.75*-0.2]
\foreach \x in {1, ..., 7}{\draw[fill] (\x - 1, 0) node[inner sep = 2pt] (\x) {$\scriptstyle \x$};}
\foreach \i\j in {1/2, 2/3, 3/4, 4/5, 5/6, 6/7}{\draw[thick] (\i) to[out = 35, in = 145] (\j);}
\end{tikzpicture}
\qquad\text{we have}\qquad
Q_{\lambda} = (1765432) = 7123456.
%\stackrel{1}{7} \stackrel{2}{1} \stackrel{3}{2} \stackrel{4}{3} \stackrel5{4} \stackrel{6}{5} \stackrel{7}{6}
\]
Considering the noncrossing partition shown in Equation~\eqref{eq:noncrossingpartitionexample} gives a more complicated example: with
\[
\lambda = \begin{tikzpicture}[scale = 0.75, baseline = 0.75*-0.2]
\foreach \x in {1, ..., 7}{\draw[fill] (\x - 1, 0) node[inner sep = 2pt] (\x) {$\scriptstyle \x$};}
\foreach \i\j in {5/6, 3/5,2/7}{\draw[thick] (\i) to[out = 35, in = 145] (\j);}
\end{tikzpicture}
\qquad\text{we have}\qquad
Q_{\lambda} = (1)(27)(365)(4) = 1764352.
%\stackrel{1}{1} \stackrel{2}{7} \stackrel{3}{6} \stackrel{4}{4} \stackrel{5}{3} \stackrel{6}{5} \stackrel{7}{2}.
\]

\begin{rem}
\label{rem:QSVnoncrossing}
The set $\QSV_{n}$ is a particular case of a more general phenomenon involving non-crossing partitions and permutations.  
Given any $n$-cycle $c \in S_{n}$,~\cite{Baine} gives a bijection between $\NCP_{n}$ and the interval between the identity and $c$ in the absolute order on $S_{n}$, and our construction realizes this bijection for $c = (1n\cdots 2)$.  
\end{rem}

%%%%%%%%%%%%%%%
\subsection{The Bruhat order on $\QSV_{n}$}
\label{sec:bruhatballot}
%%%%%%%%%%%%%%%

The Bruhat order restricts to a partial order on the set $\QSV_{n}$.
This turns out to be a very natural order, as is described in the paper~\cite{GobetWilliams}.  
This section will recall the description from this source for use in later sections.

For $n \ge 0$, define a partial order $\preceq$ on the set $\NCP_{n}$ of noncrossing partitions as the extension of the covering relation: $\lambda$ is covered by $\mu$ if and only if $\lambda$ is obtained from $\mu$ in one of the following ways:
\begin{enumerate}
\item removing an arc of the form $ \edge{i}{i+1}$ from $\mu$, or

\item replacing any arc $ \edge{i}{k}$ in $\mu$ with two arcs $ \edge{i}{j}$ and $ \edge{j}{k}$ for some $i < j < k$ which do not intersect or share a left or right endpoint with any other arc in $\mu$.

\end{enumerate}

It is difficult to describe the non-covering relations of $\preceq$ on $\NCP_{n}$---and of the Bruhat order on $\QSV_{n}$---in a direct and intuitive manner.  Instead, these relations are best understood through an intermediary object.  A \emph{ballot sequence} of length $2n$ is a sequence $b \in \{\pm1\}^{2n}$ for which each partial sum of $b$ is nonnegative and the final sum is $0$.  A well-known bijection between noncrossing partitions is used in~\cite[Section~5.1]{GobetWilliams}: for $\lambda \in \NCP_{n}$ define a ballot sequence $b^{\lambda} = (b^{\lambda}_{1}, b^{\lambda}_{2}, \ldots, b^{\lambda}_{2n})$  by
\[
b^{\lambda}_{2 k - 1} = \begin{cases} 1 & \text{if $k \notin \lambda^{-}$} \\ -1 & \text{if $k \in \lambda^{-}$} \end{cases}
\qquad
\text{and}
\qquad
b^{\lambda}_{2 k} = \begin{cases} 1 & \text{if $k \in \lambda^{+}$} \\ -1 & \text{if $k \notin \lambda^{+}$,} \end{cases}
\]
for each $1 \le k \le n$.

\begin{prop}[{\cite[Theorem 1.1 and Corollary 7.5]{GobetWilliams}}]
\label{prop:QSVorderbijection}
Let $\lambda$ and $\mu$ be noncrossing partitions of size $n$.  The following are equivalent:
\begin{enumerate}
\item $\lambda \preceq \mu$, 

\item $Q_{\lambda} \le Q_{\mu}$ in the Bruhat order, and

\item for all $1 \le k \le 2n$, $\sum_{i = 1}^{k} b^{\lambda}_{k} \le \sum_{i = 1}^{k} b^{\mu}_{k}$.

\end{enumerate}
\end{prop}

As an example, for $n = 3$ the (isomorphic) orders on $\QSV_{n}$, $\NCP_{n}$, and ballot sequences are shown in Figure~\ref{fig:Hassediagrams}.

\begin{figure}
\begin{center}
\begin{tikzpicture}
\node at (0, 4.5) (13b2) {$321$};
\node at (0, 3) (123) {$312$};
\node at (1.5, 1.5) (1b23) {$132$};
\node at (-1.5, 1.5) (12b3) {$213$};
\node at (0, 0) (1b2b3) {$123$};
\draw[thick] (1b2b3) -- (1b23);
\draw[thick] (1b2b3) -- (12b3);
\draw[thick] (1b23) -- (123);
\draw[thick] (12b3) -- (123);
\draw[thick] (123) -- (13b2);
\end{tikzpicture}
\hspace{0.75cm}
\begin{tikzpicture}
\node at (0, 4.5) (13b2) {\tikz[scale = 0.75, baseline = 0.75*-0.2]{
\foreach \x in {1, ..., 3}{\draw[fill] (\x - 1, 0) node[inner sep = 2pt] (\x) {$\scriptstyle \x$};}
\foreach \i\j in {1/3}{\draw[thick] (\i) to[out = 35, in = 145] (\j);}}};
\node at (0, 3) (123) {\tikz[scale = 0.75, baseline = 0.75*-0.2]{
\foreach \x in {1, ..., 3}{\draw[fill] (\x - 1, 0) node[inner sep = 2pt] (\x) {$\scriptstyle \x$};}
\foreach \i\j in {1/2, 2/3}{\draw[thick] (\i) to[out = 35, in = 145] (\j);}}};
\node at (1.5, 1.5) (1b23) {\tikz[scale = 0.75, baseline = 0.75*-0.2]{
\foreach \x in {1, ..., 3}{\draw[fill] (\x - 1, 0) node[inner sep = 2pt] (\x) {$\scriptstyle \x$};}
\foreach \i\j in {2/3}{\draw[thick] (\i) to[out = 35, in = 145] (\j);}}};
\node at (-1.5, 1.5) (12b3) {\tikz[scale = 0.75, baseline = 0.75*-0.2]{
\foreach \x in {1, ..., 3}{\draw[fill] (\x - 1, 0) node[inner sep = 2pt] (\x) {$\scriptstyle \x$};}
\foreach \i\j in {1/2}{\draw[thick] (\i) to[out = 35, in = 145] (\j);}}};
\node at (0, 0) (1b2b3) {\tikz[scale = 0.75, baseline = 0.75*-0.2]{
\foreach \x in {1, ..., 3}{\draw[fill] (\x - 1, 0) node[inner sep = 2pt] (\x) {$\scriptstyle \x$};}
\foreach \i\j in {}{\draw[thick] (\i) to[out = 35, in = 145] (\j);}}};
\draw[thick] (1b2b3) -- (1b23);
\draw[thick] (1b2b3) -- (12b3);
\draw[thick] (1b23) -- (123);
\draw[thick] (12b3) -- (123);
\draw[thick] (123) -- (13b2);
\end{tikzpicture}
\hspace{0.75cm}
\begin{tikzpicture}
\node at (0, 4.5) (13b2) {$111---$};
\node at (0, 3) (123) {$11-1--$};
\node at (1.5, 1.5) (1b23) {$1-11--$};
\node at (-1.5, 1.5) (12b3) {$11--1-$};
\node at (0, 0) (1b2b3) {$1-1-1-$};
\draw[thick] (1b2b3) -- (1b23);
\draw[thick] (1b2b3) -- (12b3);
\draw[thick] (1b23) -- (123);
\draw[thick] (12b3) -- (123);
\draw[thick] (123) -- (13b2);
\end{tikzpicture}
\end{center}
\caption{The Hasse diagrams of: $\QSV_{3}$ with the Bruhat order; $\NCP_{3}$ with $\preceq$; and ballot sequences (for which each $-1$ is represented as $-$) with the term-wise order on partial sums.}
\label{fig:Hassediagrams}
\end{figure}

\begin{rem}
There are two superficial differences between our presentation and that of~\cite{GobetWilliams}.
\begin{enumerate}
\item The results of~\cite{GobetWilliams} describe the Bruhat ordering of the set $\{w^{-1} \;|\; w \in \QSV_{n}\}$, rather than $\QSV_{n}$.  
In the terminology of Remark~\ref{rem:QSVnoncrossing}, these are the noncrossing partitions associated to the cycle $(12\ldots n)$ rather than $(1n\ldots 2)$.  
Inversion give an automorphism of the Bruhat order, so the results are equivalent.

\item The results of~\cite{GobetWilliams} use Dyck paths rather than ballot sequences for item 3.~in Propostition~\ref{prop:QSVorderbijection}.  
One can translate between the two by interchanging each $1$ in a ballot sequence with an up step in a Dyck path, and likewise each $-1$ with a down step.

\end{enumerate}
%
%
%Strictly speaking, the paper~\cite{GobetWilliams} phrases the above result in terms of  Dyck paths ordered by containment rather than ballot sequences ordered by partial sum.  In this context it is readily seen that the partial orders in Propostition~\ref{prop:QSVorderbijection} are graded, distributive lattices.  
%One can easily translate between ballot sequences and Dyck paths by replacing each $1$ in a ballot sequence with an up step in a Dyck path, and similarly each $-1$ with a down step.  
\end{rem}

The final result of the section follows easily from the results of~\cite{GobetWilliams}, but is not stated explicitly.  For the sake of completeness, a proof is included.

\begin{cor}
\label{cor:bruhatncphelper}
Let $\lambda$ and $\mu$ be noncrossing partitions of size $n$.  Then $\lambda \preceq \mu$ if and only if 
\[
|\lambda^{+} \cap [k-1]| - |\lambda^{-} \cap [k]| \le |\mu^{+} \cap [k-1]| - |\mu^{-} \cap [k]|
\]
and
\[
|\lambda^{+} \cap [k]| - |\lambda^{-} \cap [k]| \le |\mu^{+} \cap [k]| - |\mu^{-} \cap [k]| 
\]
for all $1 \le k \le n$.
\end{cor}
\begin{proof}
By Proposition~\ref{prop:QSVorderbijection}, it is sufficient to show that for $1 \le k \le n$,
\[
\sum_{i = 1}^{2k - 1} b^{\lambda}_{i}  = 1 + 2|\lambda^{+} \cap [k-1]| - 2|\lambda^{-} \cap [k]|
\qquad\text{and}\qquad
\sum_{i = 1}^{2k} b^{\lambda}_{i}  = 2|\lambda^{+} \cap [k]| - 2|\lambda^{-} \cap [k]|.
\] 
This can be established inductively: for $k = 1$ both equations clearly hold, and for $k > 1$, we consider the differences between the $k-1$st expression and the $k$th: 
\[
(1 + 2|\lambda^{+} \cap [k-1]| - 2|\lambda^{-} \cap [k]|) - (2|\lambda^{+} \cap [k-1]| + 2|\lambda^{-} \cap [k-1]|)
= 1 - 2|\lambda^{-} \cap \{k\}|,
\]
which is $b^{\lambda}_{2k-1}$, and 
\[
(2|\lambda^{+} \cap [k]| - 2|\lambda^{-} \cap [k]|) - (1 + 2|\lambda^{+} \cap [k-1]| + 2|\lambda^{-} \cap [k]|) = 2|\lambda^{+} \cap \{k\}| - 1,
\]
which is $b^{\lambda}_{2k}$.
\end{proof}

\begin{rem}
Under the straightforward bijection between Dyck paths and partition diagram under the staircase, the order in Proposition~\ref{prop:QSVorderbijection}-3 is exactly the dual of the Young lattice 
of partition diagrams under the staircase.
\end{rem}
%
%\newpage
%
%
%%This section will describe that order combinatorially.
%
% and accordingly, induces a partial order on the set $\NCP_{n}$, 
%
%
%
%.  Define a partial order $\preceq$ on the set of noncrossing partitions by
%\[
%\lambda \preceq \mu 
%\qquad\text{if and only if}\qquad
%Q_{\lambda} \le Q_{\mu}.
%\]
%
%\begin{prop}
%\label{prop:ncpcovering}
%Let $n \ge 0$ and take $\lambda, \mu \in \NCP$.  Then $\mu$ covers $\lambda$ in $\preceq$ if and only if $\lambda$ is obtained from $\mu$ by one of the following procedures:
%\begin{enumerate}
%\item removing one arc of the form $\edge{i}{i+1}$ from $\mu$, or
%
%\item replacing an arc $\edge{i}{k}$ in $\mu$ with two arcs $\edge{i}{j}, \edge{j}{k}$ for some $i < j < k$ such that the resulting arcs do not intersect any others in $\mu$.
%
%\end{enumerate}
%\end{prop}
%
%The proof of Proposition~\ref{prop:ncpcovering} follows some explanatory remarks.  As an example of the result, the Hasse diagram for $\preceq$ on $\NCP_{3}$ is:
%
%
%The proof of the proposition involves encoding elements of $\QSV_{n}$ and $\NCP_{n}$ as a third Catalan object known as a \emph{ballot sequence}: a sequence $b \in \{\pm 1\}^{2n}$ which sums to $0$, and for which each partial sum $b_{1} + b_{2} + \cdots + b_{k}$ is nonnegative.  Ballot sequences are ordered by
%\[
%b \preceq c \qquad\text{if and only if}\qquad \sum_{i=1}^{k} b_{i} \le \sum_{i = 1}^{k} c_{i} \;\text{for all $1 \le k \le 2n$},
%\]
%for which the downward covering relation given by replacing a sequential subsequence $(-1, 1)$ by $(1, -1)$, provided that the result is still a ballot sequence.  
%
%Given a noncrossing partition $\lambda \in \NCP_{n}$, define a ballot sequence $b^{\lambda} = (b^{\lambda}_{1}, b^{\lambda}_{2}, \ldots, b^{\lambda}_{2n})$ by
%\[
%b^{\lambda}_{2 k - 1} = \begin{cases} 1 & \text{if $k \notin \lambda^{-}$} \\ -1 & \text{if $k \in \lambda^{-}$} \end{cases}
%\qquad
%\text{and}
%\qquad
%b^{\lambda}_{2 k} = \begin{cases} 1 & \text{if $k \in \lambda^{+}$} \\ -1 & \text{if $k \notin \lambda^{+}$,} \end{cases}
%\]
%%\[
%%(b^{\lambda}_{2 k - 1}, b^{\lambda}_{2 k}) = \begin{cases}
%%(1, 1) & \text{$k \in \lambda^{+}$ and $k \notin \lambda^{-}$} \\
%%(1, -1) & \text{$k \notin \lambda^{+}$ and $k \notin \lambda^{-}$}  \\
%%(-1, 1) & \text{$k \in \lambda^{+}$ and $k \in \lambda^{-}$}  \\
%%(-1, -1) & \text{$k \notin \lambda^{+}$ and $k \in \lambda^{-}$} 
%%\end{cases}
%%\]
%for each $1 \le k \le n$. Thus,
%\[
%\sum_{i = 1}^{2k - 1} b_{k}^{\lambda} = 1 + 2(|\lambda^{+} \cap [k-1]| - |\lambda^{-} \cap [k]|)
%\qquad\text{and}\qquad
%\sum_{i = 1}^{2k} b_{k}^{\lambda} = 2(|\lambda^{+} \cap [k]| - |\lambda^{-} \cap [k]|).
%\]
%
%\begin{lem}
%\label{lem:ballotbijection}
%The map $\lambda \shortmapsto b^{\lambda}$ is a bijection, and the covering relations on ballot sequences correspond to the covering relations in described in Proposition~\ref{prop:ncpcovering}.
%\end{lem}
%\begin{proof}
%Lemma~\ref{lem:noncrossingpartitionproperty} and case-checking.
%\end{proof}
%
%Thus, it is sufficient to check that $Q_{\lambda} \le Q_{\mu}$ if and only if $b^{\lambda} \preceq b^{\mu}$.
%
%\begin{lem}
%\label{lem:ballotimpliesbruhat}
%Let $\lambda$ and $\mu$ be noncrossing partitions of size $n$ for which $b^{\lambda}$ is covered by $b^{\mu}$ in $\prec$.  Then $Q_{\lambda} \le Q_{\mu}$ in the Bruhat order.
%\end{lem}
%\begin{proof}
%Check using characterization in Proposition~\ref{prop:ncpcovering} and Lemma~\ref{lem:ballotbijection} above.  Covering relation type 1.~is simply removing a simple transposition $(i, i+1)$, and covering relation type 2. is replacing a (non-simple) transposition $(i, k)$ with the product $(i, j)(j, k)$, which is a subword. 
%\end{proof}
%
%In the remaining results, recall the Tableau $\TTT(w)$ for $w \in S_{n}$ as in Section~\ref{sec:bruhat}, whose $n-k$th row is denoted by $\TTT_{k}(w)$.
%
%
%\begin{lem}
%\label{lem:tableautoballot}
%Let $\lambda \in \NCP_{n}$.  Then for $1 \le k \le n$,
%\[
%\sum_{i = 1}^{2k - 1} b_{k}^{\lambda} = 1 + 2 |\{j \in [n] \setminus \TTT_{k}(Q_{\lambda}) \;|\; j < k \}|
%\qquad\text{and}\qquad
%\sum_{i = 1}^{2k} b_{k}^{\lambda} = 2 |\{j \in \TTT_{k}(Q_{\lambda}) \;|\; j > k \}|.
%\]
%\end{lem}
%\begin{proof}
%It is sufficient to show that
%\[
%|\{j \in [n] \setminus \TTT_{k}(Q_{\lambda}) \;|\; j < k \}| = |\lambda^{+} \cap [k-1]| - |\lambda^{-} \cap [k]|
%\]
%and
%\[
% |\{j \in \TTT_{k}(Q_{\lambda}) \;|\; j > k \}| =  |\lambda^{+} \cap [k]| - |\lambda^{-} \cap [k]|.
%\]
%Both sides of the first equation count arcs crossing from $[k-1]$ and $[n] \setminus [k]$, and the both sides of the second count arcs between $[k]$ and $[n] \setminus [k]$.
%\end{proof}
%
%The next result is stated in slightly greater generality than needed in this section, so as to be applied in later sections as well.
%
%\begin{prop}
%\label{prop:bruhatimpliesballot}
%Let $\lambda \in \NCP_{n}$.  For any permutation $w \in S_{n}$ which precedes $Q_{\lambda}$ in the Bruhat order (so $w \le Q_{\lambda}$), we have:
%\[
%|\{j \in [n] \setminus \TTT_{k}(w) \;|\; j < k \}| \le |\{j \in [n] \setminus \TTT_{k}(Q_{\lambda}) \;|\; j < k \}|
%\]
%and 
%\[
%|\{j \in \TTT_{k}(w) \;|\; j > k \}| \le |\{j \in \TTT_{k}(Q_{\lambda}) \;|\; j > k \}|.
%\]
%\end{prop}
%\begin{proof}
%Follows from Tableau Criterion.
%\end{proof}
%
%\begin{proof}[Proof of Proposition~\ref{prop:ncpcovering}]
%By Lemma~\ref{lem:ballotbijection}, is is sufficient to show that $Q_{\lambda} \le Q_{\mu}$ if and only if $b^{\lambda} \preceq b^{\mu}$.  The ``if'' follows from Lemma~\ref{lem:ballotimpliesbruhat}, while the ``only if'' follows from Proposition~\ref{prop:bruhatimpliesballot}.
%\end{proof}

%%%%%%%%%%%%%%%%%%%%%%%%%%%%%%%%%%%%%%%%
\section{The excedance quotient of the Bruhat order}
\label{sec:excedance}
%%%%%%%%%%%%%%%%%%%%%%%%%%%%%%%%%%%%%%%%

This section will describe a quotient of the Bruhat order on $S_{n}$ under an equivalence relation defined by the weak excedances of a permutation.  
This quotient order has a number of desirable properties, which are also explored in the section.  
One particularly noteworthy result is that each equivalence class contains a unique element of the set $\QSV_{n}$ defined in Section~\ref{sec:QSV}, so the construction of this order can be seen as dual to the results of~\cite{GobetWilliams} described in Section~\ref{sec:bruhatballot}.  

Given a permutation $w \in S_{n}$, a \emph{weak excedance} of $w$ is a pair $(i, w_{i})$ for which $i \le w_{i}$.  Disaggregating, define the \emph{excedance values} $\EV(w)$ and \emph{excedance positions} $\EP(w)$ to be
\begin{align*}
\EV(w) &= \{ w_{i} \;|\; \text{$(i, w_{i})$ is a weak excedance of $w$}\},\;\text{and} \\[0.5em]
\EP(w) &= \{ i  \;|\; \text{$(i, w_{i})$ is a weak excedance of $w$}\}.
\end{align*}
Excedances and the sets $\EV(w)$ and $\EP(w)$ are easiest seen using two-line notation for permutations.  For example, marking non-excedancs in red,
\[
w = \overset{1}{3} \overset{2}{5} {\color{red} \overset{3}{1}} \overset{4}{4} {\color{red} \overset{5}{2}} \overset{6}{6} {\color{red} \overset{7}{5}} \overset{8}{8},
\qquad
\EP(w) = \{1, 2, 4, 6, 8\},\qquad\text{and}\qquad
\EV(w) = \{3, 4, 5, 6, 8\}.
\]
We define the \emph{excedance relation} $\sim$ on $S_{n}$ by:
\begin{equation}
\label{eq:excednacerel}
v \sim w \qquad\text{if and only if} \qquad \text{$\EV(v) = \EV(w)$ and $\EP(v) = \EP(w)$},
\end{equation}
and say that each equivalence class of $S_{n}/\sim$ is an \emph{excedance class}.

This section will investigate the properties of excedance classes and their interaction with the Bruhat order; a summary of the main results follows.  Each noncrossing partition $\lambda$ of size $n$ determines an excedance class: recall the sets $\lambda^{+}$ and $\lambda^{-}$ defined in Section~\ref{sec:ncp}, and let
\[
\CCC_{\lambda} = \{ w \in S_{n} \;|\;  \text{$\EV(w) = [n] \setminus \lambda^{+}$ and $\EP(w) = [n] \setminus \lambda^{-}$}  \}.
\]
In Section~\ref{sec:ex1}, it is shown that every  excedance class of $S_{n}$ is of the form $\CCC_{\lambda}$ for some noncrossing partition.  As an example, the are five excedance classes of $S_{3}$ are given below:
\[
\mathcal{C}_{\begin{tikzpicture}[scale = 0.35, baseline = 0.35*-0.2]
\foreach \x in {1, ..., 3}{\draw[fill] (\x - 1, 0) circle (2pt) node[inner sep = 2pt] (\x) {};}
\foreach \x in {1, ..., 3}{\node[below] at (\x) {$\scriptstyle \x$};}
\foreach \i\j in {1/3}{\draw[thick] (\i) to[out = 35, in = 145] (\j);}
\end{tikzpicture}} = \{ \stackrel{1}{3}  \stackrel{2}{2} \color{red} \stackrel{3}{1} \color{black}, \stackrel{1}{2}\stackrel{2}{3} \color{red} \stackrel{3}{1} \color{black} \}, \qquad
\mathcal{C}_{\begin{tikzpicture}[scale = 0.35, baseline = 0.35*-0.2]
\foreach \x in {1, ..., 3}{\draw[fill] (\x - 1, 0) circle (2pt) node[inner sep = 2pt] (\x) {};}
\foreach \x in {1, ..., 3}{\node[below] at (\x) {$\scriptstyle \x$};}
\foreach \i\j in {1/2, 2/3}{\draw[thick] (\i) to[out = 35, in = 145] (\j);}
\end{tikzpicture}} = \{ \stackrel{1}{3} \color{red} \stackrel{2}{1} \stackrel{3}{2}  \color{black} \}, \qquad
\mathcal{C}_{\begin{tikzpicture}[scale = 0.35, baseline = 0.35*-0.2]
\foreach \x in {1, ..., 3}{\draw[fill] (\x - 1, 0) circle (2pt) node[inner sep = 2pt] (\x) {};}
\foreach \x in {1, ..., 3}{\node[below] at (\x) {$\scriptstyle \x$};}
\foreach \i\j in {1/2}{\draw[thick] (\i) to[out = 35, in = 145] (\j);}
\end{tikzpicture}} = \{\stackrel{1}{2}  \color{red} \stackrel{2}{1} \color{black}  \stackrel{3}{3}\}, 
\]
\[
\mathcal{C}_{\begin{tikzpicture}[scale = 0.35, baseline = 0.35*-0.2]
\foreach \x in {1, ..., 3}{\draw[fill] (\x - 1, 0) circle (2pt) node[inner sep = 2pt] (\x) {};}
\foreach \x in {1, ..., 3}{\node[below] at (\x) {$\scriptstyle \x$};}
\foreach \i\j in {2/3}{\draw[thick] (\i) to[out = 35, in = 145] (\j);}
\end{tikzpicture}} = \{\stackrel{1}{1} \stackrel{2}{3}  \color{red} \stackrel{3}{2} \color{black} \}, \qquad\text{and}\qquad
\mathcal{C}_{\begin{tikzpicture}[scale = 0.35, baseline = 0.35*-0.2]
\foreach \x in {1, ..., 3}{\draw[fill] (\x - 1, 0) circle (2pt) node[inner sep = 2pt] (\x) {};}
\foreach \x in {1, ..., 3}{\node[below] at (\x) {$\scriptstyle \x$};}
%\foreach \i\j in {2/3}{\draw[thick] (\i) to[out = 35, in = 145] (\j);}
\end{tikzpicture}} = \{\stackrel{1}{1} \stackrel{2}{2} \stackrel{3}{3} \}.
\]
The Bruhat order descends to a relation on excedance classes, and we show that this relation is a partial order; in particular Section~\ref{sec:ex3} proves the following result.  Recall the order $\preceq$ from Section~\ref{sec:bruhatballot}.

\begin{thm}
\label{thm:excedancequotient}
Writing $\le$ for the partial order on excedance classes $S_{n}/\!\!\sim$ induced by the Bruhat order, $\CCC_{\lambda} \le \CCC_{\mu}$ if and only if $\lambda \preceq \mu$.
\end{thm}

Two important intermediate results stated in Section~\ref{sec:ex2} show that each excedance class $\CCC_{\lambda}$ contains a unique minimal and maximal elements, which are respectively a $321$-avoiding permutation and the element $Q_{\lambda} \in \QSV_{n}$.  Combined with Theorem~\ref{thm:excedancequotient}, this gives the following corollary.  

\begin{cor}\label{cor:interval}
Each excedance class $\CCC_{\lambda}$ is an interval in the Bruhat order, with upper bound $Q_{\lambda}$ and lower bound given by a $321$-avoiding permutation.
\end{cor}

%%%%%%%%%%%%%%%
\subsection{Excedance classes and noncrossing partitions}
\label{sec:ex1}
%%%%%%%%%%%%%%%

This section will establish some basic results which relate excedance classes to noncrossing partitions.  

\begin{prop}
For $n \ge 0$, the map
\[
\begin{array}{ccc}
\NCP_{n} & \longrightarrow & S_{n}/\!\!\sim \\
\lambda & \longmapsto & \CCC_{\lambda}
\end{array}
\]
is a bijection.
\end{prop}
\begin{proof}
We will first show that every excedance class of $S_{n}$ has the form $\CCC_{\lambda}$ for some $\lambda \in \NCP_{n}$.  
This is equivalent to showing that for any $w \in S_{n}$, the criterion of Lemma~\ref{lem:noncrossingpartitionproperty} holds:
\[
|\{i \in [k-1] \;|\; i \notin \EV(w)\}| \ge |\{ j \in [k] \;|\; j \notin \EP(w)\}| \qquad\text{for all $1 \le k \le n$}.
\]
We will establish the above inequality directly.  Fix $1 \le k \le n$ and suppose that $i \in [k]$ is not an excedance position, so that that $i > w_{i}$.  By assumption, $w_{i} < k$ and $w_{i}$ is not an excedance value.  Thus, 
\[
\{w_{i} \;|\; \text{$i \in [k-1]$ and $i \notin \EV(w)$} \} \subseteq \{j \in [k] \;|\; j \notin \EP(w)\},
\]
giving the claim.  Now, we must show that each $\CCC_{\lambda}$ is nonempty.  This is established by Lemma~\ref{lem:QSVexcedance} below, which completes the proof.
\end{proof}


\begin{lem}
\label{lem:QSVexcedance}
For any noncrossing partition $\lambda$, $Q_{\lambda} \in \CCC_{\lambda}$.  In particular, if $Q_{\lambda} \in S_{n}$, 
\[
\EV(Q_{\lambda}) = [n] \setminus \lambda^{+}
\qquad\text{and}\qquad
\EP(Q_{\lambda}) = [n] \setminus \lambda^{-}.
\]
\end{lem}
\begin{proof}
By definition, $Q_{\lambda}(j) < j$ if and only if the arc $ \edge{Q_{\lambda}(j)}{j}$ appears in $\lambda$, in which case $j \in \lambda^{-}$ and $Q_{\lambda}(j) \in \lambda^{+}$; this establishes the claim.
\end{proof}

The final result relates certain properties of the elements of $\CCC_{\lambda}$ to the partial sums of the ballot sequence $b^{\lambda}$ defined in Section~\ref{sec:bruhatballot}, by way of Corollary~\ref{cor:bruhatncphelper}.  This will be key to a number of arguments in subsequent sections.


\begin{lem}
\label{lem:tableautobruhat}
Let $\lambda$ be a noncrossing partition of size $n \ge 0$ and take $w \in \CCC_{\lambda}$.  For all $1 \le k \le n$, 
\[
|\{ w_{i} \;|\; \text{$n \ge i > k$ and $w_{i} < k$}\}|
%|\{j \in [n] \setminus \TTT_{k}(w) \;|\; j < k \}| 
= |\lambda^{+} \cap [k-1]| - |\lambda^{-} \cap [k]|
\]
and 
\[
|\{ w_{i} \;|\; \text{$1 \le i \le k$ and $w_{i} > k$}\}|
%|\{j \in \TTT_{k}(w) \;|\; j > k \}| 
= |\lambda^{+} \cap [k]| - |\lambda^{-} \cap [k] |.
\]
\end{lem}
\begin{proof}
To show the first equation, note that by definition
\[
|\lambda^{+} \cap [k-1]| - |\lambda^{-} \cap [k]| = |\{j \in [k-1] \;|\; j \notin \EV(w)\}| - |\{i \in [k] \;|\; i \notin \EP(w)\}|.
\]
For $i \in [k]$ with $i \notin \EP(w)$, it must be the case that $w_{i} < i$ and so $w_{i} \in [k-1]$ with $w_{i} \notin \EV(w)$.  
Therefore, the equation above counts the $w_{i} \in [k-1]$ for which $w_{i} \notin \EV(w)$ and $i \notin [k]$; this is exactly $\{ w_{i} \;|\; \text{$n \ge i > k$ and $w_{i} < k$}\}$.

The second equation follows from a similar but somewhat more complicated argument.  We begin by manipulating the right side into a more suitable form:
\begin{align*}
|\lambda^{+} \cap [k]| - |\lambda^{-} \cap [k] | &=  (k - |\lambda^{-} \cap [k] |) - (k - |\lambda^{+} \cap [k]|) \\
&= |\EP(w) \cap [k]| - |\EV(w) \cap [k]|.
\end{align*}
Now, for $j \in \EV(w) \cap [k]$, we have $j = w_{i}$ for some $i \le j$, so that $i \in \EP(w) \cap [k]$.  Since $i \in \EP(w) \cap [k]$ implies that $w_{i} \in \EV(w)$, the equation above counts the positions $i \in \EP(w) \cap [k]$ for which $w_{i} \notin [k]$; this set is equinumerous to $\{ w_{i} \;|\; \text{$1 \le i \le k$ and $w_{i} > k$}\}$.
\end{proof}

%%%%%%%%%%%%%%%
\subsection{Minimal and maximal elements}%Excedance classes are Bruhat intervals}
\label{sec:minmax}
\label{sec:ex2}
%%%%%%%%%%%%%%%

This section will show that each excedance class contains a unique Bruhat minimum and maximum.  
This is a key intermediate step to showing that excedance classes are Bruhat intervals with a well-defined quotient order.  
%This section will show that each excedance class is an interval in the Bruhat order, and in doing so explicitly specify the maximal and minimal elements.  
We will begin with the maximal elements, which are the elements of $\QSV_{n}$, while the minimal ($321$-avoiding) elements are discussed in Subsection~\ref{sec:ex2.1}

\begin{prop}
\label{prop:QSVinterval}
For all noncrossing partitions $\lambda$, $Q_{\lambda}$ is the Bruhat maximum element of $\CCC_{\lambda}$.  
%In particular $w \le Q_{\lambda}$ for all $w \in \CCC_{\lambda}$.
\end{prop}
\begin{proof}
Lemma~\ref{lem:QSVexcedance} shows that $Q_{\lambda} \in \CCC_{\lambda}$, so we need only show that $Q_{\lambda}$ is an upper bound for $\CCC_{\lambda}$; to this end, fix $w \in \CCC_{\lambda}$.  Using the tableau criterion (Proposition~\ref{TableauCriterion}), it is sufficient to show that $\TTT_{k}(w)$ is entry-wise less than or equal to $\TTT_{k}(Q_{\lambda})$ for each $1 \le k \le n$.

The argument consists of two distinct parts, first conducting an element-by-element comparison of the entries of $\TTT_{k}(w)$ and $\TTT_{k}(Q_{\lambda})$ which are strictly greater than $k$, and then doing the same for the elements which are at most $k$.  The validity of this hinges on the fact that these collections of entires have the same cardinality for $w$ and $Q_{\lambda}$: by Lemma~\ref{lem:tableautobruhat}, 
\begin{align*}
|\{ w_{i} \;|\; \text{$1 \le i \le k$ and $w_{i} > k$}\}|
&= |\{ Q_{\lambda}(i) \;|\; \text{$1 \le i \le k$ and $w_{i} > k$}\}| \\
&= |\lambda^{+} \cap [k]| - |\lambda^{-} \cap [k] |,
\end{align*}
and consequently $|\{ w_{i} \;|\; \text{$1 \le i \le k$ and $w_{i} \le k$}\}| = |\{ Q_{\lambda}(i) \;|\; \text{$1 \le i \le k$ and $w_{i} \le k$}\}|$.

We begin with the first part.  Enumerate the entries of $\TTT_{k}(w)$ and $\TTT_{k}(Q_{\lambda})$ which are greater than $k$ in increasing order as $x_{1} < x_{2} < \cdots <  x_{r}$ and $q_{1} < q_{2} < \ldots < q_{r}$.  We aim to show that $x_{i} \le q_{i}$ for each $1 \le i \le r$.  
Fixing one such $i$, Lemma~\ref{lem:tableautobruhat} gives that
\[
|\{ w_{t} \;|\; \text{$1 \le t \le q_{i}$ and $w_{t} > q_{i}$} \}| = |\lambda^{+} \cap [q_{i}]| - |\lambda^{-} \cap [q_{i}]|.
\]
%and each of $x_{1}, x_{2}, \ldots, x_{r}$ is contained in the set on the left.  
It is sufficient to show that the above quantity is equal to $r - i$, as this implies that each of $x_{1}, \ldots x_{i}$ must be less than $q_{i}$.

Let $C_{i}$ denote the connected component of $\lambda$ containing $y_{i}$.  
From the definition of $Q_{\lambda}$, $q_{i}$ must be the maximal element of $C_{i}$,  and $Q_{\lambda}^{-1}(q_{i})$ the minimal element.  
Combinatorially, the difference $|\lambda^{+} \cap [q_{i}]| - |\lambda^{-} \cap [q_{i}]|$ counts the number of arcs in $\lambda$ with left endpoint in $[q_{i}]$ and right endpoint in $[n] \setminus [q_{i}]$, and as $q_{i} \in \EV(w)$, this is the number of arcs in $\lambda$ which are above $q_{i}$.  
Every arc in $\lambda$ which lies above $q_{i}$ must fully contain $C_{i}$ between its left and right endpoints, and so in particular, the left endpoint of such an arc is contained in $[k]$, and the right endpoint is greater than $q_{i}$.  
Thus, each such arc belongs to the connected component of one of the elements $q_{i+1}, \ldots, q_{r}$; there are precisely $r - i$ such connected components.

For the second part of the argument, we aim to show that the entries of $\TTT_{k}(w)$ which are at most $k$ are entry-wise less than or equal to the analogous entries of $\TTT_{k}(Q_{\lambda})$, and we establish this in an indirect manner described below.  
Writing $s = |\lambda^{+} \cap [k-1]| - |\lambda^{-} \cap [k]|$, Lemma~\ref{lem:tableautobruhat} states that there are exactly $s$ elements of $[k]$ which do not appear in $\TTT_{k}(w)$, and likewise for $\TTT_{k}(Q_{\lambda})$.  
Respectively enumerate these elements in increasing order as $x_{1} < x_{2} < \cdots < x_{s}$ and $q_{1} < q_{2} < \cdots < q_{s}$.  
We will show that $q_{i} \le x_{i}$ for each $1 \le i \le s$, as this implies the opposite comparison for the remaining elements of $[k] \setminus \{q_{1}, q_{2}, \ldots, q_{s}\}$ and $[k] \setminus \{x_{1}, x_{2}, \ldots, x_{s}\}$ as desired.  

Fixing $1 \le i \le s$, Lemma~\ref{lem:tableautobruhat} gives that
\[
|\{ w_{t} \;|\; \text{$n \ge t > k$ and $w_{t} < q_{i}$} \}| = |\lambda^{+} \cap [q_{i} - 1]| - |\lambda^{-} \cap [q_{i}]|.
\]
It is therefore sufficient to show that the above quantity $i-1$, so that $q_{i}$ is bounded above by each of $x_{i}, x_{i+1}, \ldots, x_{s}$.

Combinatorially, the difference $|\lambda^{+} \cap [q_{i} - 1]| - |\lambda^{-} \cap [q_{i}]|$ counts the number of arcs in $\lambda$ with a left endpoint $[q_{i} -1]$ and a right endpoint in $[n] \setminus [q_{i}]$, or equivalently, the arcs above $q_{i}$ in $\lambda$.  
Writing $C_{i}$ for the connected component containing $q_{i}$, each such arc must contain $C_{i}$ between its left and right endpoints, so the left endpoint is less that $q_{i}$ and the right endpoint lies somewhere in $[n] \setminus [k]$.  
Thus, each such arc must belong to the connected component of one of $q_{1}, q_{2}, \ldots, q_{i-1}$, and there are $i-1$ such connected components.
\end{proof}

%%%%%%%%%%%%%%%
\subsubsection{The minimal element of an excedance class}
\label{sec:ex2.1}
%%%%%%%%%%%%%%%

We now turn to the minimal element of each excedance class described in Theorem~\ref{thm:excedancequotient}.  For a noncrossing partition $\lambda$ of size $n$, write 
\[
\lambda^{+} = \{a_{1} < a_{2} < \cdots < a_{s}\},
\qquad
\lambda^{-} = \{b_{1} < b_{2} < \cdots < b_{s}\},
\]
\[
[n] \setminus \lambda^{+} = \{x_{1} < x_{2} < \cdots < x_{n-s}\},
\qquad\text{and}\qquad
[n] \setminus \lambda^{-} = \{y_{1} < y_{2} < \cdots < y_{n-s}\},
\]
so that the elements of each set are enumerated in increasing order.  Let $T_{\lambda} \in S_{n}$ be the permutation with
\[
T_{\lambda}(i) = \begin{cases} a_{s} & \text{if $i = b_{s}$} \\ x_{s} & \text{if $i = y_{s}$.}  \end{cases}
\]
Thus, the one-line notation for $T_{\lambda}$ can be obtained by placing the elements of $\lambda^{+}$ in increasing left-to-right order in the positions $\lambda^{-}$, and placing the remaining elements of $[n]$ in the remaining positions in the same manner.  For example, with $n = 8$ and 
\[
\lambda = \begin{tikzpicture}[scale = 0.75, baseline = 0.75*-0.2]
\foreach \x in {1, ..., 8}{\draw[fill] (\x - 1, 0) node[inner sep = 2pt] (\x) {$\scriptstyle \x$};}
\foreach \i\j in {1/5, 2/3, 5/7}{\draw[thick] (\i) to[out = 35, in = 145] (\j);}
\end{tikzpicture}
\]
we have $\lambda^{+} = \{1, 2, 5\}$ and $\lambda^{-} = \{3, 5, 7\}$, $[8] \setminus \lambda^{+} = \{3, 4, 6, 7, 8\}$, and $[8] \setminus \lambda^{-} = \{1, 2, 4, 6, 8\}$, and consequently
\[
T_{\lambda} = \overset{1}{3} \overset{2}{4} {\color{red} \overset{3}{1}} \overset{4}{6} {\color{red} \overset{5}{2}} \overset{6}{7} {\color{red} \overset{7}{5}} \overset{8}{8},
\]
where non-excedances are marked in red, as at the beginning of Section~\ref{sec:excedance}.

\begin{prop}
\label{prop:321avoid}
For all noncrossing partitions $\lambda$, $T_{\lambda} \in \CCC_{\lambda}$, and this is the Bruhat-minimum element of $\CCC_{\lambda}$.
\end{prop}
\begin{proof}
To see that $T_{\lambda} \in \CCC_{\lambda}$, recall the elements $a_{i}, b_{i}, x_{i}$, and $y_{i}$ defined above for $\lambda$. For $1 \le r \le s$,
\[
a_{r} = \min\{ k  \;|\;  r \ge |[k] \cap \lambda^{+}| \}
\qquad\text{and}\qquad
b_{r} = \min\{ k  \;|\;  r \ge |[k] \cap \lambda^{-}| \}.
\]
By Lemma~\ref{lem:noncrossingpartitionproperty}, it is always the case that $|[k] \cap \lambda^{-}| \le |[k-1] \cap \lambda^{+}|$, and so $b_{r} < a_{r}$.  Thus the $(b_{r}, a_{r})$ is not a weak excedence of $T_{\lambda}$.  
A similar argument shows that every pair $(y_{r}, x_{r})$ is a weak excedance, so that 
\[
\lambda^{+} = [n] \setminus \EV(T_{\lambda})
\qquad\text{and}\qquad
\lambda^{-} =  [n] \setminus \EP(T_{\lambda}).
\]

To see that $T_{\lambda} \le w$ for all $w \in \CCC_{\lambda}$, recall the Tableau Criterion, Proposition~\ref{TableauCriterion}.  
For $1 \le k \le n$, the row $\TTT_{k}(T_{\lambda})$ will consist of the $|[k] \cap \lambda^{-}|$ smallest elements of $\EV(w)$ along with the $k - |[k] \cap \lambda^{-}|$ smallest elements of $[n] \setminus \EV(w)$.  
For $w$,  the row $\TTT_{k}(w)$ will also consist of $|[k] \cap \lambda^{-}|$ elements of $\EV(w)$ and $k - |[k] \cap \lambda^{-}|$ elements of $[n] \setminus \EV(w)$, but these elements need not be the minimal ones.  
Thus, by assumption of minimality, $\TTT_{k}(T_{\lambda})$ is entry-wise less than or equal to $\TTT_{k}(w)$.
\end{proof}

For the next result, recall that a permutation $w \in S_{n}$ is \emph{$321$-avoiding} if there do not exist indices $i < j < k$ for which $w_{i} > w_{j} > w_{k}$.  The number of $321$-avoiding permutations is known to be the $n$th Catalan number, so the following result establishes that each excedance class contains a unique $321$-avoiding permutation.

\begin{prop}
For all noncrossing partitions $\lambda$, the permutation $T_{\lambda}$ is $321$-avoiding.
\end{prop}
\begin{proof}
Let $i, j, k\in [n]$ and assume without loss of generality that $i < j < k$.  Since any element of $[n]$ must be contained in either $\lambda^{+}$ or its complement, we must have two elements of $\{i, j, k\}$ which belong to one of $\lambda^{-}$ or $[n] \setminus \lambda^{-}$.  Since $T_{\lambda}$ restricts to an order-preserving bijection from $\lambda^{+}$ to $\lambda^{-}$, and from $[n] \setminus \lambda^{+}$ to $[n] \setminus \lambda^{-}$, this implies that no $321$-pattern can exist in $T_{\lambda}$, giving the first claim.
\end{proof}

%%%%%%%%%%%%%%%
\subsection{Comparing excedance classes}
\label{sec:ex3}
%%%%%%%%%%%%%%%

This section will prove Theorem~\ref{thm:excedancequotient}, following an intermediate result.

To begin, recall the order $\preceq$ on noncrossing partitions defined in Section~\ref{sec:bruhatballot}

\begin{prop}
\label{prop:Qdominates}
Let $\mu$ be a noncrossing partition of size $n$.  Then
\[
\{w \in S_{n} \;|\; w \le Q_{\mu}\} = \bigsqcup_{\lambda \preceq \mu} \CCC_{\lambda}.
\]
\end{prop}
\begin{proof}
To begin, assume that $\lambda \preceq \mu$, so that by Proposition~\ref{prop:QSVorderbijection}, $Q_{\lambda} \le Q_{\mu}$.  By Proposition~\ref{prop:QSVinterval}, we have $w \le Q_{\lambda}$ for any $w \in C_{\lambda}$, so we have
\[
\{w \in S_{n} \;|\; w \le Q_{\mu}\} \supseteq \bigsqcup_{\lambda \preceq \mu} \CCC_{\lambda}.
\]

To see the opposite containment, suppose that $w$ is a permutation with $w \le Q_{\mu}$ and let $\lambda$ be the unique noncrossing partition for which $w \in \CCC_{\lambda}$.  
By way of Corollary~\ref{cor:bruhatncphelper}, $\lambda \preceq \mu$ is equivalent to the inequalities
\[
|\lambda^{+} \cap [k-1]| - |\lambda^{-} \cap [k]| \le |\mu^{+} \cap [k-1]| - |\mu^{-} \cap [k]|
\]
and
\[
|\lambda^{+} \cap [k]| - |\lambda^{-} \cap [k]| \le |\mu^{+} \cap [k]| - |\mu^{-} \cap [k]|
\]
for all $1 \le k \le n$.  
We will establish this equivalent formulation of our claim using the characterization of each side given in Lemma~\ref{lem:tableautobruhat}.  
From the assumption that $w \le Q_{\mu}$, the tableau criterion (Proposition~\ref{TableauCriterion}) states that the tableau $\TTT(w)$ is entry-wise less than or equal to $\TTT(Q_{\mu})$.  
Thus, for each $1 \le k \le n$, 
\[
|\{w_{i} \;|\; \text{$1 \le i \le k$ and $w_{i} > k$} \}|
\le
|\{Q_{\mu}(i) \;|\; \text{$1 \le i \le k$ and $Q_{\mu}(i) > k$} \}|,
\]
since each entry of $\TTT_{k}(w)$ is bounded above by the corresponding entry of $\TTT_{k}(Q_{\mu})$, and likewise
\[
|\{ w_{i} \;|\; \text{$n \ge i > k$ and $w_{i} < k$} \}| 
\le
|\{ Q_{\mu}(i) \;|\; \text{$n \ge i > k$ and $Q_{\mu}(i) < k$} \}|,
\]
since each entry of $\TTT_{k}(Q_{\mu})$ is bounded below by the corresponding entry of $\TTT_{k}(w)$.  This completes the proof.
\end{proof}

We now prove Theorem~\ref{thm:excedancequotient}.  Recall the elements $T_{\lambda}$ defined in Section~\ref{sec:ex2.1}.

\begin{proof}[Proof of Theorem~\ref{thm:excedancequotient}]
Define a relation on excedance classes:
\[
\CCC_{\lambda} \le \CCC_{\mu}
\qquad\text{if and only if}\qquad
\text{$v \le w$ for some $v \in \CCC_{\lambda}$ and $w \in \CCC_{\mu}$}.
\]
As Lemma~\ref{lem:QSVexcedance} states that $Q_{\lambda} \in \CCC_{\lambda}$ for each noncrossing partition $\lambda$, Proposition~\ref{prop:QSVorderbijection} implies that $\CCC_{\lambda} \le \CCC_{\mu}$ whenever $\lambda \preceq \mu$.  It is therefore sufficient to show the converse: $\CCC_{\lambda} \le \CCC_{\mu}$ only if $\lambda \preceq \mu$.  To this end, suppose that $\CCC_{\lambda} \le \CCC_{\mu}$, so that $v \le w$ for some $v \in \CCC_{\lambda}$ and $w \in \CCC_{\mu}$.  With Proposition~\ref{prop:QSVinterval}, this implies that $v \le Q_{\lambda}$.  Then by Proposition~\ref{prop:Qdominates}, we have that $\lambda \preceq \mu$, completing the proof.
\end{proof}

%%%%%%%%%%%%%%%
\subsection{The weak order on excedance classes}
%%%%%%%%%%%%%%%

Theorem~\ref{thm:excedancequotient} and Corollary~\ref{cor:interval} show that the excedance classes partition the Bruhat order of $S_n$ in intervals and the quotient $S_n/\!\!\sim$ is a 
well defined order isomorphic to lattice $\preceq$ on noncrossing partitions. Another relevant order on $S_n$ is the (left) weak order generated by the  covering relation
	$$u\lessdot_w v \quad\iff\quad  \begin{cases} \ell(v)=\ell(u)+1, \text{ and } \\ vu^{-1} \text{ is a simple transposition } (a\, a+1) \end{cases}$$
This order is a suborder of the Bruhat order and there are several reason to study the weak order of $S_n$. For example, see~\cite{A,B,C} and much more. 
We are thus interested to use the quotient we define in Theorem~\ref{thm:excedancequotient} to define a weak order on excedance classes.
This will give us a weak order on the set $QSV_n$ which will be of particular interest when we will study the points of $QSV_n$ as a variety.  
In a sequel article, we will describe further interesting properties of the weak order. Here we shall define it and mention some of its properties in subsequent sections.

The (left) weak ordering on $S_n/\!\!\sim$ is generated by the  covering relation
  $$ C_{\lambda} \lessdot_w C_{\mu} \quad\iff\quad  Q_\lambda \lessdot_w  v\in C_{\mu} \text{ for some } v.$$
 This is a well defined suborder of the Bruhat order on $S_n/\!\!\sim$
 
 \begin{rem} 
  One can also consider the weak order restricted to the set $QSV_n$ as we did in Section~\ref{sec:bruhatballot}.
  Unlike the Bruhat order, the two construction are not isomorphic. The order on the quotient is much more amenable to our
  current and future work than the restriction to $QSV_n$. We leave to the interested reader the study of the weak order
  restricted to $QSV_n$.
 \end{rem}


%%%%%%%%%%%%%%%%%%%%%%%%%%%%%%%%%%%%%%%%
\section{Bases for the Temperley--Lieb Algebra $\TL_{n}(2)$}
%%%%%%%%%%%%%%%%%%%%%%%%%%%%%%%%%%%%%%%%

The Temperley--Lieb algebra $\TL_{n}(2)$ is the $\CC$-algebra generated by elements $e_{1}, \ldots, e_{n-1}$ subject to the following relations for each $1 \le i, j \le n$
\[
\begin{array}{rll}
e_{i}^{2} &= 2 e_{i} \\
e_{i}e_{j} &= e_{j}e_{i} & \text{if $|i - j| > 1$} \\
e_{i} e_{j} e_{i} &= e_{i} & \text{if $|i - j| = 1$}.
\end{array}
\]
There is a subjective algebra morphism from the symmetric group algebra $\CC S_{n}$ to $\TL_{n}(2)$ given by 
\[
\begin{array}{rcl}
\phi: \CC S_{n} & \longrightarrow & \TL_{n}(2) \\
s_{i} & \longmapsto & 1 - e_{i}.
\end{array}
\]
In particular $\TL_{n}(2)\cong S_n/\ker(\phi)$.

It is well-known that the images of all $321$-avoiding permutations under $\phi$ forms a basis for $\TL_{n}(2)$.  
Another basis, due to Zinno~\cite{Zinno} can be obtained via the map $\phi$ using the combinatorics of noncrossing partitions described in Remark~\ref{rem:QSVnoncrossing}.  This basis is precisely the set $\{\phi(w^{-1}) \;|\; w \in \QSV_{n}\}$.  Since the kernel
\[
\ker(\phi) = \langle (13) - (123) - (132) + (12) + (23) - e \rangle
\]
is invariant under the $\CC$-linear map $w \mapsto w^{-1}$, the following result is an immediate consequence of~\cite[Theorem 2]{Zinno}.

\begin{thm}
\label{thm:TLbasis}
For all $n \ge 0$, the set $\phi(\QSV_{n})$ is a basis for $\TL_{n}(2)$.
\end{thm}

\begin{rem}
Theorem~\ref{thm:TLbasis} also follows from the results of Gobet and Williams in~\cite{GobetWilliams} as a special case of a much stronger result which gives a basis for $\TL_{n}(2)$ for each Coxeter element of $S_{n}$.  
However,~\cite{GobetWilliams} does not provide an explicit statement of this result, so it is easier to deduce the theorem as a consequence of Zinno's work directly.
\end{rem}

%%%%%%%%%%%%%%%%%%%%%%%%%%%%%%%%%%%%%%%%
\subsection{More Bases for the Temperley--Lieb Algebra $\TL_{n}(2)$}
%%%%%%%%%%%%%%%%%%%%%%%%%%%%%%%%%%%%%%%%

Using excedance classes, we discovered that one can give a more general theorem for constructing Temperley--Lieb Algebra basis from permutation.
We include it here with its proof as it is a nice result of our current investigation.

\begin{thm}
\label{thm:TLbases}
Let $n \ge 0$ and for each $\lambda \in \NCP_{n}$, fix an element $w_{\lambda} \in \CCC_{\lambda}$.  Then the set 
$\{\phi(w_{\lambda}) \;|\; \lambda \in \NCP_{n}\}$ is a basis of $\TL_{n}(2)$.
\end{thm}

The Theorem~\ref{thm:TLbasis} then follow as a corollary of Theorem~\ref{thm:TLbases}.
In general, however, many bases obtained via Theorem~\ref{thm:TLbases} are novel.  A minimal example can be obtained with $n = 4$: the set 
\[
\{
\stackrel{1}{4} \stackrel{2}{3} \color{red} \stackrel{3}{1} \stackrel{4}{2} \color{black}, 
\stackrel{1}{4} \stackrel{2}{2} \stackrel{3}{3}  \color{red} \stackrel{4}{1} \color{black}, 
\stackrel{1}{4} \stackrel{2}{2}\color{red}  \stackrel{3}{1} \stackrel{4}{3} \color{black}, 
\stackrel{1}{3} \color{red}  \stackrel{2}{1} \color{black} \stackrel{3}{4} \color{red} \stackrel{4}{2}\color{black} , 
\stackrel{1}{1} \stackrel{2}{4} \stackrel{3}{3} \color{red} \stackrel{4}{2} \color{black}, 
\stackrel{1}{4} \color{red} \stackrel{2}{1} \stackrel{3}{2} \stackrel{4}{3} \color{black}, 
\stackrel{1}{3} \stackrel{2}{2} \color{red} \stackrel{3}{1} \color{black} \stackrel{4}{4}, 
\stackrel{1}{3} \color{red} \stackrel{2}{1} \stackrel{3}{2} \color{black} \stackrel{4}{4}, 
\stackrel{1}{2} \color{red} \stackrel{2}{1} \color{black} \stackrel{3}{4} \color{red} \stackrel{4}{3} \color{black}, 
\stackrel{1}{1} \stackrel{2}{3} \color{red} \stackrel{3}{2} \stackrel{4}{3} \color{black}, 
\stackrel{1}{2}  \color{red} \stackrel{2}{1} \color{black} \stackrel{3}{3} \stackrel{4}{4}, 
\stackrel{1}{1} \stackrel{2}{3} \color{red} \stackrel{3}{2} \color{black} \stackrel{4}{4}, 
\stackrel{1}{1} \stackrel{2}{2} \stackrel{3}{4}  \color{red} \stackrel{4}{3} \color{black},
\stackrel{1}{1} \stackrel{2}{2} \stackrel{3}{3} \stackrel{4}{4}
\}
\]
meets the criteria of Theorem~\ref{thm:TLbases}, and accordingly maps to a basis of $\TL_{n}(2)$ under $\phi$. 
This set is neither $\QSV_{4}$ not the set of $321$-avoiding permutations ($4312 \notin \QSV_{4}$ and is not $321$-avoiding).
Moreover, the set above is not described in~\cite{GobetWilliams, Zinno}: each subset of $S_{4}$ in these sources which is not $\phi(\QSV_{n})$ contains more than one element from certain excedance classes and none from others.  


%%%%%%%%%%%%%%%%%%%%%%%%%%%%%%%%%%%%%%%%
\subsection{A presentation of the Temperley--Lieb algebra}
\label{sec:TLpres}
%%%%%%%%%%%%%%%%%%%%%%%%%%%%%%%%%%%%%%%%

Using the isomorphism $\TL_{n}(2)\cong S_n/\ker(\phi)$ is equivalent to applying a certain $\CC$-linear relations on the natural basis of $\CC S_{n}$.  
Recall that a $321$-pattern in a permutation $w \in S_{n}$ is a triple $i < j < k$ for which $w_{i} > w_{j} > w_{k}$.  Given such a pattern, we can write
\[
w = \mathbf{a}w_{i}\mathbf{b}w_{j}\mathbf{c}w_{k}\mathbf{d},
\]
where $\mathbf{a}$, $\mathbf{b}$, $\mathbf{c}$, and $\mathbf{d}$ are (possibly empty) subwords of $w$.  The quotient $\CC S_{n} / \ker(\phi)$ is then described by the relations
\begin{equation}
\label{eq:321relation}
w \equiv \mathbf{a}w_{j}\mathbf{b}w_{i}\mathbf{c}w_{k}\mathbf{d} + \mathbf{a}w_{i}\mathbf{b}w_{k}\mathbf{c}w_{j}\mathbf{d} - \mathbf{a}w_{j}\mathbf{b}w_{k}\mathbf{c}w_{i}\mathbf{d} - \mathbf{a}w_{k}\mathbf{b}w_{i}\mathbf{c}w_{j}\mathbf{d} + \mathbf{a}w_{k}\mathbf{b}w_{j}\mathbf{c}w_{i}\mathbf{d}
\end{equation}
for each $321$-pattern in each permutation $w \in S_{n}$.

\begin{lem}
Let $w \in S_{n}$ be a permutation with a $321$-pattern in positions $i < j < k$.  Then $w > w'$ for each $w'$ in the set
\[
\{
\mathbf{a}w_{j}\mathbf{b}w_{i}\mathbf{c}w_{k}\mathbf{d},  \,
\mathbf{a}w_{i}\mathbf{b}w_{k}\mathbf{c}w_{j}\mathbf{d}, \,
\mathbf{a}w_{j}\mathbf{b}w_{k}\mathbf{c}w_{i}\mathbf{d}, \,
\mathbf{a}w_{k}\mathbf{b}w_{i}\mathbf{c}w_{j}\mathbf{d}, \,
\mathbf{a}w_{k}\mathbf{b}w_{j}\mathbf{c}w_{i}\mathbf{d}
\}
\]
\end{lem}
\begin{proof}
Let $w'=\mathbf{a}w_{j}\mathbf{b}w_{i}\mathbf{c}w_{k}\mathbf{d}$, we have $w(w')^{-1}= (w_j\,w_i)$ is a transposition. 
Every inversion of $w'$ is an inversion on $w$. On the other hand, $w$ has the inversion $i<j$, $w_i>w_j$, and potentially more (two for each $b\in \mathbf{b}$ where $w_j<b<w_i$).
Hence $\ell(w')<\ell(w)$, and therefor $w>w'$ (see~\cite[Chapter 2]{BjornerBrenti}). All other cases are similar, one transposition at a time.
\end{proof}

\nantel{I am here}

%%%%%%%%%%%%%%%%%%%%%%%%%%%%%%%%%%%%%%%%
\subsection{Proof of Theorem~\ref{thm:TLbases}}
%%%%%%%%%%%%%%%%%%%%%%%%%%%%%%%%%%%%%%%%

This section proves Theorem~\ref{thm:TLbases}, which follows from the triangularity established in the next result.  Recall the $321$-avoiding permutation $T_{\lambda}$ defined in Section~\ref{} for each noncrossing partition $\lambda \in \NCP_{n}$.

\begin{prop}
\label{prop:TLbases}
For $n \ge 0$, take $w \in S_{n}$ and let $\lambda \in \NCP_{n}$ be the unique noncrossing partition for which $w \in \CCC_{\lambda}$.  Then
\[
w \equiv T_{\lambda} + \sum_{\mu \prec \lambda} a_{\mu}^{w} T_{\mu} \qquad \text{(mod $ \ker\phi$)}
\]
for some coefficients $a_{\mu}^{w} \in \ZZ$.
\end{prop}

A proof of the Proposition follows the next Lemma.

\begin{lem}
\label{lem:321excedance}
Suppose that $w \in S_{n}$ has a $321$-pattern.  Then $w$ has a $321$-pattern $i < j < k$ with $i \in \EP(w)$ and $k \notin \EP(w)$, and moreover exactly one element of the set 
\[
\{
\mathbf{a}w_{j}\mathbf{b}w_{i}\mathbf{c}w_{k}\mathbf{d},  \,
\mathbf{a}w_{i}\mathbf{b}w_{k}\mathbf{c}w_{j}\mathbf{d}, \,
\mathbf{a}w_{j}\mathbf{b}w_{k}\mathbf{c}w_{i}\mathbf{d}, \,
\mathbf{a}w_{k}\mathbf{b}w_{i}\mathbf{c}w_{j}\mathbf{d}, \,
\mathbf{a}w_{k}\mathbf{b}w_{j}\mathbf{c}w_{i}\mathbf{d}
\}
\]
belongs to the same excedance class as $w$.
\end{lem}
\begin{proof}
We begin with the first claim.  
By assumption, there exists an $321$-pattern $i_{0} < j_{0} < k_{0}$ in $w$.  
It must be the case that either $i_{0} \in \EP(w)$ or $k_{0} \notin \EP(w)$, as the first implies that $w_{k_{0}} < w_{j_{0}} < w_{i_{0}} < i_{0} < j_{0} < k_{0}$, and the second that $i_{0} < j_{0} < k_{0} \le w_{k_{0}} < w_{j_{0}} < w_{i_{0}}$.
If $i_{0} \in \EP(w)$, the resulting inequality implies that there exists an element $i < i_{0}$ for which $w_{i} \ge i_{0}$; as a consequence we obtain a $321$-pattern $i < j_{0} < k_{0}$ in $w$ with $i \in \EP(w)$ and $k_{0} \notin \EP(w)$.  
Following a similar argument, if $k_{0} \in \EP(w)$, then there exists an element $k > k_{0}$ for which $w_{k} \le k_{0}$; we then have a $321$-pattern $i_{0} < j_{0} < k$ for which $k \notin \EP(w)$ and $i_{0} \in \EP(w)$.

For the second argument, we write $i < j < k$ for the $321$-pattern described in the Lemma; there are two cases, in which $j \in \EP(w)$ or not.  We will consider the first case, as the second follows from a similar argument.  In this case, it is straightforward to see that each element $w'$ in the set
\[
\{
\mathbf{a}w_{i}\mathbf{b}w_{k}\mathbf{c}w_{j}\mathbf{d}, \,
\mathbf{a}w_{j}\mathbf{b}w_{k}\mathbf{c}w_{i}\mathbf{d}, \,
\mathbf{a}w_{k}\mathbf{b}w_{i}\mathbf{c}w_{j}\mathbf{d}, \,
\mathbf{a}w_{k}\mathbf{b}w_{j}\mathbf{c}w_{i}\mathbf{d}
\}
\]
will have a difference excedance class: either $k$ is an excedance position of $w'$, in which case $\EP(w') \neq \EP(w)$, or  or one of $w_{i}$ or $w_{j}$ is a not excedance value for $w'$, in which case $\EV(w') \neq \EV(w)$.  What remains is to show that with $w'' = \mathbf{a}w_{j}\mathbf{b}w_{i}\mathbf{c}w_{k}\mathbf{d}$, 
\[
\EP(w'') = \EP(w)
\qquad\text{and}\qquad
\EV(w'') = \EV(w).
\]
For all $s \in [n] \setminus \{i, j\}$, we have $w''_{i} = w_{i}$, so the above claim reduces to $i, j \in \EP(w')$ and $w_{i}, w_{j} \in \EV(w'')$.  To show this, we observe that $i < j \le w_{j} < w_{i}$, and so $w''_{i} = w_{j} > i$ and $w''_{j} = w_{i} > j$.
\end{proof}

\begin{proof}[Proof of Proposition~\ref{prop:TLbases}]
We proceed by induction on the Bruhat order of $S_{n}$.  If $w$ is $321$-avoiding, then $w = T_{\lambda}$ and the claim clearly holds.  Assuming that $w$ is not $321$-avoiding, we have $T_{\lambda} \le w$, and so we assume for the sake of induction that for each $v \le w$ the claim holds.  As $w$ has a $321$-pattern, we apply Lemma~\ref{lem:321excedance}, which along with Equation~\eqref{eq:321relation} gives an expression
\[
w \equiv w'' + \sum_{\mu \prec \lambda} \sum_{w' \in \CCC_{\mu}} b_{w'} w',
\]
where $w''$ is an element of $\CCC_{\lambda}$ with $w'' \le w$, each coefficient $b_{w'} \in \{1, 0, -1\}$, and all but five of the $b_{w'}$ are zero.  Using Proposition~\ref{prop:Qdominates}, each $w'$ in the sum above also precedes $w$ in the Burhat order, so the inductive hypothesis applies, giving
\[
w \equiv T_{\lambda} + \sum_{\mu \prec \lambda} a_{\mu}^{w'} T_{\mu} + \sum_{\mu \prec \lambda} \sum_{w' \in \CCC_{\mu}} \left( b_{w'} T_{\mu} + \sum_{\nu \prec \mu} b_{w'} a_{\nu}^{w''} T_{\nu}\right).
\]
This completes the proof.
\end{proof}


%%%%%%%%%%%%%%%%%%%%%%%%%%%%%%%%%%%%%%
%%%%%%%%%%%%%%%%%%%%%%%%%%%%%%%%%%%%%%
%%%%%%%%%%%%%%%%%%%%%%%%%%%%%%%%%%%%%%
\section{Quasisymmetric vanishing polynomials}
In this section we define polynomials in $n$ variables that vanishes on the set $QSV_n$ and such that
the homogeneous top degree is quasisymmetric. More precisely, let $\alpha=(\alpha_1,\alpha_2,\ldots,\alpha_k)$
be any fixed composition of $d>0$. Let 
	$$M_\alpha(x_1,x_2,\ldots,x_n)=\sum_{1\le i_1<i_2<\cdots<i_k\le n} x_{i_1}^{\alpha_1} x_{i_2}^{\alpha_2}\cdots  x_{i_k}^{\alpha_k}$$
be the quasisymmetric monomial indexed by $\alpha$. We will define a polynomial $P_\alpha(x_1,x_2,\ldots,x_n)$ such that
  $P_\alpha = M_\alpha +$  lower degree terms, and  $P_\alpha(\sigma)=0$  for all $\sigma\in QSV_n$.

$1=f_1<f_2<\cdots <f_\ell<f_{\ell+1}=n+1$ and 
$\beta_i=\alpha_{f_i}+\alpha_{f_i+1}+\cdots+\alpha_{f_{i+1}-1}$.
When $\alpha$ refine $\beta$, we write $\alpha\le\beta$.

\begin{definition}\label{def:vanishP}
 For any $\alpha=(\alpha_1,\alpha_2,\ldots,\alpha_k)$, the polynomial $P_\alpha(x_1,x_2,\ldots,x_n)$ is defined as
	$$P_\alpha=\sum_{\ell=1}^k (-1)^{k-\ell} \sum_{1=f_1<f_2<\cdots <f_{\ell+1}=k+1  \atop 1\le i_1<i_2<\cdots < i_\ell\le n} \prod_{j=1}^\ell 
	      \Big( (x_{i_j}^{\alpha_{f_j}}- i_j^{\alpha_{f_j }})  i_j^{\alpha_{f_j +1}+\cdots+\alpha_{f_{j +1}-1}} \Big)
	      $$
	      The top degree of $P_\alpha$ is in the sum when $\ell=k$, we must have $f_i=i$ in this case. Choosing the variable $x_{i_j}$ in all binomials $(x_{i_j}^{\alpha_{f_j}}- i_j^{\alpha_{f_j }})$, we get $M_\alpha$.
\end{definition}
	      
For a fix $\sigma$, we will partition the sums in $P_\alpha$ according to the cycle type of $\sigma$ and the non-crossing structure
of these cycle will play a major role in showing the following theorem.

\begin{thm}\label{thm:vanishing}
 For any $\alpha$ and any $\sigma\in QSV_n$ we  have $P_\alpha(\sigma)=0$.
\end{thm}

\begin{proof}
 Let $\sigma=C_1C_2\cdots C_r$  the decomposition of $\sigma$ into disjoint cycle. We include the fix points a 1-cycles.
 Given a set on indices $1\le i_1<i_2<\cdots < i_\ell\le n$ we say that the cycle support of $S=\{ i_1,i_2,\ldots , i_\ell\}$
 is $C(S)=\{j : S\cap C_j \not= \emptyset \}$. We have
 	$\displaystyle P_\alpha=\sum_{T\subseteq [r]} P_{\alpha,T},$
where 
\begin{equation}\label{eq:PT}
P_{\alpha,T}=
	\sum_{\ell=|T|}^k (-1)^{k-\ell} \hskip-.8cm \sum_{{1=f_1<f_2<\cdots <f_{\ell+1}=k+1  \atop 1\le i_1<i_2<\cdots < i_\ell\le n}\atop C(\{ i_1,i_2,\ldots , i_\ell\})=T} \prod_{j=1}^\ell 
	      \Big( (x_{i_j}^{\alpha_{f_j}}- i_j^{\alpha_{f_j }})  i_j^{\alpha_{f_j +1}+\cdots+\alpha_{f_{j +1}-1}} \Big)\,.
\end{equation}
We show that $P_{\alpha,T}(\sigma)=0$ for all $T$. If $T=\emptyset$ there is nothing to show as $P_{\alpha,\emptyset}=0$. 
We first consider the case when $|T|=1$, and then use the non-crossing structure of the cycles to reduce the case $|T|>1$
to $|T'|=1$.

\medskip
\noindent{\bf Case $\bf |T|=1$}: Let $T=\{t\}$ and $C_t=(a_m\,\ldots\,a_2\,a_1)$, where $a_1<a_2<\cdots<a_m$. Remark that on such cycle, the variable $x_{a_i}=a_{i-1}$ with the convention that $a_0=a_m$.
Expanding all the product in the definition of $P_{\alpha,T}$ and evaluating at $\sigma$, we obtain
\begin{equation}\label{eq:PTatsigma}
	P_{\alpha,\{t\}}(\sigma)=
	\sum_{{1\le j_1\le j_2\le\cdots\le j_k\le m}} \sum_{{\epsilon_1,\epsilon_2,\ldots,\epsilon_k \atop \epsilon_i\in \{0,1\}}\atop \epsilon_i=1 \text{ if } j_{i-1}=j_i} (-1)^{\sum \epsilon_i}\ 
	   z_{j_1,\epsilon_1}^{\alpha_1}z_{j_2,\epsilon_2}^{\alpha_2} \cdots z_{j_k,\epsilon_k}^{\alpha_k}\,,
\end{equation}
where $z_{j_i,\epsilon_i}=a_{j_i-1}$ if $\epsilon_i=0$, otherwise $z_{j_i,\epsilon_i}=a_{j_i}$. To show that $P_{\alpha,\{t\}}(\sigma)$, we construct a sign reversing involution 
on the evaluation of the terms in Equation~\eqref{eq:PTatsigma}.

Given $1\le j_1\le j_2\le\cdots\le j_k\le m$ and $\epsilon_1,\epsilon_2,\ldots,\epsilon_k$, then the term 
$$z_{j_1,\epsilon_1}^{\alpha_1}z_{j_2,\epsilon_2}^{\alpha_2} \cdots z_{j_k,\epsilon_k}^{\alpha_k}=a_{s_1}^{\alpha_1}a_{s_2}^{\alpha_2}\cdots a_{s_k}^{\alpha_k},$$
where $s_i\in\{j_i,j_i-1\}$ depending on $\epsilon_i$ and  $0\le s_1\le s_2\le \cdots\le s_k\le m$. First find the largest, rightmost $s_i\not\in\{0,m\}$. 

\medskip
\noindent
If $\epsilon_i=1$, then let $\epsilon'_i=0$ and $j'_i=j_i+1$, fixing all other values $\epsilon'_p=\epsilon_p$ and $j'_p=j_p$ for $p\not= i$.

\medskip
In this case remark that $z_{j_i,\epsilon_i}=a_{j_i}=z_{j'_i,\epsilon'_i}$ and $z_{j_p,\epsilon_p}=z_{j'_p,\epsilon'_p}$ for  $p\not= i$. Hence
 $$ (-1)^{\sum \epsilon'_i}\ z_{j'_1,\epsilon'_1}^{\alpha_1}z_{j'_2,\epsilon'_2}^{\alpha_2} \cdots z_{j'_k,\epsilon'_k}^{\alpha_k}
  = - (-1)^{\sum \epsilon_i}\ z_{j_1,\epsilon_1}^{\alpha_1}z_{j_2,\epsilon_2}^{\alpha_2} \cdots z_{j_k,\epsilon_k}^{\alpha_k}$$
 To show that $z_{j'_1,\epsilon'_1}^{\alpha_1}z_{j'_2,\epsilon'_2}^{\alpha_2} \cdots z_{j'_k,\epsilon'_k}^{\alpha_k}$ is a term of the sum Equation~\eqref{eq:PTatsigma}
 we need to show that $j'_1\le\cdots\le j'_{i}\le j'_{i+1}\le\cdots\le j'_{k}$ and since $\epsilon'_i=0$, we need $j'_{i-1}<j'_i$.
 For the last inequality, we have $j'_{i-1}=j_{i-1}\le j_i<j_i+1=j'_i$. For the other inequality among the $j$'s, we only need to show that $j'_{i}=j_i+1\le j_{i+1}= j'_{i+1}$.
 Here we recall that $i$ is chosen so that $s_i$ is the rightmost values such that $0<s_i<m$. This implies that either $i=k$ and there is no $j_{i+1}$ or $s_{i+1}=m\in\{j_{i+1},j_{i+1}-1\}$
 and $j_{i+1}\ge m >s_i=j_i$. The last equality follows from $\epsilon_i=1$. We thus have that all such terms cancelled each other in Equation~\eqref{eq:PTatsigma}.
 
 The case where $\epsilon_i=0$ is very similar and is the reverse of the operation above. The choice of the rightmost $s_i\not\in\{0,n\}$ will be the same in both cases, showing that we indeed
 have a sign reversing involution. All terms such that the evaluation $a_{s_1}^{\alpha_1}a_{s_2}^{\alpha_2}\cdots a_{s_k}^{\alpha_k}$ contains some $0<s_i<m$ will cancel.
 The only two terms that survive the cancelation are
 $$(-1)^{n-1}a_{0}^{\alpha_1}a_{m}^{\alpha_2}\cdots a_{m}^{\alpha_k}+  (-1)^n a_{m}^{\alpha_1}a_{m}^{\alpha_2}\cdots a_{m}^{\alpha_k}$$
 that can only be obtained when $j_1=1$ and $j_2=\cdots=j_k=m$ with $\epsilon_1=0$ and $\epsilon_2=\cdots=\epsilon_k=1$, for the first term;
 and when $j'_1=\cdots=j'_k=m$ with $\epsilon'_1=\cdots=\epsilon'_k=1$ for the second term. Since $a_0=a_m$:
  $$P_{\alpha,\{t\}}(\sigma)=(-1)^{n-1}a_{0}^{\alpha_1}a_{m}^{\alpha_2}\cdots a_{m}^{\alpha_k}+  (-1)^n a_{m}^{\alpha_1}a_{m}^{\alpha_2}\cdots a_{m}^{\alpha_k}=0.$$


\medskip 
\noindent{\bf Case $\bf |T|>1$}: 
When we have more than one cycle involved, let $T=\{t_1,t_2,\ldots,t_r\}$ and we assume (without lost of generality) that $C=C_{t_1}$
is a cycle that do not contain (in the non-crossing sense) any nested cycles among $C_{t_j}$ for $j>1$. The fact that the cycles of $\sigma$ are non-crossing guaranties the existence of such $C$
for any given $T$.  We now partition the terms of Equation~\eqref{eq:PT} according to the intersection of $ i_1<i_2<\cdots < i_\ell$ with the $C_{t_j}$ for $j>1$ and the corresponding possible choices of $f_j$'s.
We  show that the portion of the terms intersecting $C$ is a vanishing polynomials as in the case $|T|=1$. Let $c=\min(C)$ and $d=\max(C)$.
Assume we have $C(\{ i_1,i_2,\ldots , i_\ell\})=T$ and let 
 	$$CQ_j(\{i_1,i_2,\ldots,i_\ell\})=\{i_1,i_2,\ldots,i_\ell\}\cap C_{t_j}\not= \emptyset\,.$$ 
From our choice of $C=C_{t_1}$, we have $CQ_j(\{i_1,i_2,\ldots,i_\ell\})=\{i_1,i_2,\ldots,i_\ell\}\cap \{i: c\le i\le d\}$. Outside the range $[c,d]=\{i: c\le i\le d\}$, we fix all the other parameters involved in the terms
$P_{\alpha,T}$ in Equation~\eqref{eq:PT}.
Fix $Q=(Q_2,\ldots,Q_r)$ where $\emptyset\not= Q_j\subset C_{t_j}$ such that 
	$$\bigcup_{j=2}^r Q_j = \{\underline{i}_1,\underline{i}_2,\ldots,\underline{i}_p\} \cup  \{\overline{i}_1,\overline{i}_2,\ldots,\overline{i}_q\}$$
where $p+q<k$ and 
	$$\underline{i}_1<\underline{i}_2<\cdots<\underline{i}_p<c\le d<\overline{i}_1<\overline{i}_2<\cdots<\overline{i}_q\,.$$
We also fix $F=\{\underline{f}_1,\underline{f}_2,\ldots,\underline{f}_p,\underline{f}_{p+1},\overline{f}_1,\overline{f}_2,\ldots,\overline{f}_q,\overline{f}_{q+1}\}$. where
	$$1=\underline{f}_1<\underline{f}_2<\cdots<\underline{f}_p<\underline{f}_{p+1}<\overline{f}_1<\overline{f}_2<\cdots<\overline{f}_q<\overline{f}_{q+1}=k+1\,.$$
For any term of the sum in  Equation~\eqref{eq:PT}, we have a unique corresponding $Q$ and $F$. 
In particular, we have $\displaystyle P_{\alpha,T}=\sum_Q P_{\alpha,T,Q,F}$ where
\begin{align*}\label{eq:PTQ}
P_{\alpha,T,Q,F}= &\prod_{j=1}^p 
	     	\Big( (x_{\underline{i}_j}^{\alpha_{\underline{f}_j}}-\underline{i}_j^{\alpha_{\underline{f}_j }})  \underline{i}_j^{\alpha_{\underline{f}_j +1}+\cdots+\alpha_{\underline{f}_{j +1}-1}} \Big) \times\\
	&\sum_{\ell=1}^{k-p-q} (-1)^{k-p-q-\ell} \hskip-1cm
	\sum_{{{{\underline{f}_{p+1}=f_1<f_2<\cdots <f_{\ell-p-q+1}=\overline{f}_1   \atop c \le i_1<i_2<\cdots < i_{\ell-p-q} \le d}\atop C(\{ i_1,i_2,\ldots , i_{\ell-p-q}\})=\{t_1\}}}}
		\prod_{j=1}^\ell 
	     	\Big( (x_{i_j}^{\alpha_{f_j}}- i_j^{\alpha_{f_j }})  i_j^{\alpha_{f_j +1}+\cdots+\alpha_{f_{j +1}-1}} \Big)\times\\
		&\quad\prod_{j=1}^q 
	     	\Big( (x_{\overline{i}_j}^{\alpha_{\overline{f}_j}}-\overline{i}_j^{\alpha_{\overline{f}_j }})  \overline{i}_j^{\alpha_{\overline{f}_j +1}+\cdots+\alpha_{\overline{f}_{j +1}-1}} \Big) \,.
\end{align*}
When we evaluate $P_{\alpha,T,Q,F}(\sigma)$, then centred term above is $P_{(\alpha_{\underline{f}_{p+1}}, \alpha_{\underline{f}_{p+1}+1} , \ldots, \alpha_{\overline{f}_1} ),\{t_1\}}(C_{t_1})=0$ using the result for $|\{t_1\}|=1$ in first part of the proof. This complete the proof.
\end{proof}



\newpage

\begin{thebibliography}{99}




\bibitem{AB} J. C. Aval, N. Bergeron,
\textit{Catalan paths and quasi-symmetric functions}.
Proc. of the Am. Math. Soc., 2003, 131(4), pp. 1053--1062.
\href{https://doi.org/10.1090/S0002-9939-02-06634-0}{10.1090/S0002-9939-02-06634-0}.

\bibitem{ABB} J. C. Aval, F. Bergeron, N. Bergeron,
\textit{Ideals of quasi-symmetric functions and super-covariant polynomials for $S_n$}.
Adv. in Math., 2004, 181 (2), pp. 353--367.
\href{https://doi.org/10.1016/S0001-8708(03)00068-9}{10.1016/S0001-8708(03)00068-9}.

\bibitem{Baine}  Philippe Biane.  \textit{Some properties of crossings and partitions}. Discrete Mathamatics {\bf 175} (1997) 41-53.

\bibitem{BjornerBrenti}  Francesco Brenti and Anders Bj\"{o}rner.  Combinatorics of Coxeter Groups. Germany: Springer Berlin Heidelberg, 2010.


\bibitem{CLO} D. Cox, J. Little, D. O'Shea,
\textit{Ideals, varieties, and algorithms: an introduction to computational
algebraic geometry and commutative algebra}.
Springer Science \& Business Media; 2013 Mar 9.

\bibitem{GobetWilliams}  Thomas Gobet and Nathan Williams.  \textit{Noncrossing partitions and Bruhat order}. European Journal of Combinatorics {\bf 53} (2016) 8-34.


\bibitem{L} S. X. Li,
\textit{Ideals and quotients of diagonally quasi-symmetric functions}.
Elec. J. Comb., Vol 24, Issue \#3, P3.3.
\href{https://doi.org/10.37236/6658}{10.37236/6658}.

\bibitem{MacSchub} I. G. Macdonald,
\textit{Notes on Schubert polynomials}.
Publications LACIM, vol. 6, Universit\'e du Qu\'ebec \`a Montr\'eal,  (1991) [ISBN 978-2-89276-086-6].



\bibitem{Stanley} R. Stanley,
\textit{Enumerative combinatorics. {V}ol. 2}.
{Cambridge Studies in Advanced Mathematics}, Vol. {62},
{Cambridge University Press, Cambridge}, (1999). [and its {Catalan Addendum} https://math.mit.edu/$\sim$rstan/ec/catadd.pdf]

\bibitem{Zinno}  Matthew Zinno.  \textit{A Temperley--Lieb basis coming from the Braid group}. Journal of Knot Theory and Its Ramifications 11, {\bf 4} (2002) 575-599.



\end{thebibliography}



\end{document}


