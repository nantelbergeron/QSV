\documentclass[11pt,oneside]{amsart}

\usepackage[usenames,dvipsnames,svgnames,table]{xcolor}
\usepackage[colorlinks=true, pdfstartview=FitV, linkcolor=blue, citecolor=blue, urlcolor=darkblue]{hyperref}

%\usepackage{addfont}
%\addfont{OT1}{rsfs10}{\rsfs}

\usepackage{microtype}
\usepackage{geometry}                % See geometry.pdf to learn the layout options. There are lots.
\geometry{letterpaper}                   % ... or a4paper or a5paper or ...
%\geometry{landscape}                % Activate for for rotated page geometry
%\usepackage[parfill]{parskip}    % Activate to begin paragraphs with an empty line rather than an indent
\usepackage{graphicx}
\usepackage{mathrsfs}
\usepackage{amssymb}
\usepackage{amsfonts}
\usepackage{mathrsfs}
\usepackage{epstopdf}
\usepackage{lscape}
\usepackage[utf8]{inputenc}
\usepackage{tikz,caption}
\DeclareGraphicsRule{.tif}{png}{.png}{`convert #1 `dirname #1`/`basename #1 .tif`.png}
\usepackage{enumitem}
\setlist[itemize]{leftmargin=2em}
\setlist[enumerate]{leftmargin=2em}
\usepackage{booktabs}
\usepackage{multirow}
\usepackage{mathtools}
\usepackage[linesnumbered,ruled]{algorithm2e}

\definecolor{darkblue}{rgb}{0.0,0,0.7} % darkblue color
\definecolor{darkred}{rgb}{0.7,0,0} % darkred color
\definecolor{darkgreen}{rgb}{0, .6, 0} % darkgreen color

% Dark red emphasis
\newcommand{\defncolor}{\color{darkred}}
\newcommand{\defn}[1]{{\defncolor\emph{#1}}} % emphasis of a definition

\newtheorem{theorem}{Theorem}[section]
\newtheorem{prop}[theorem]{Proposition}
\newtheorem{cor}[theorem]{Corollary}
\newtheorem{lemma}[theorem]{Lemma}
\newtheorem{conj}[theorem]{Conjecture}
\theoremstyle{definition}
\newtheorem{definition}[theorem]{Definition}
\newtheorem{example}[theorem]{Example}
\newtheorem{remark}[theorem]{Remark}
\numberwithin{equation}{section}

\usepackage[colorinlistoftodos]{todonotes}
\newcommand{\idiot}[1]{\vspace{5 mm}\par \noindent
\marginpar{\textsc{Note}}
\framebox{\begin{minipage}[c]{0.95 \textwidth}
#1 \end{minipage}}\vspace{5 mm}\par}
\newcommand{\mike}[1]{\todo[size=\tiny,color=lime!30]{#1 \\ \hfill --- Mike}}
\newcommand{\nantel}[1]{\todo[size=\tiny,color=red!30]{#1 \\ \hfill --- Nantel}}
\newcommand{\farad}[2][]{\todo[size=\tiny,color=ForestGreen!30,#1]{#2 \\ \hfill --- Farad}}
\newcommand{\kelvin}[1]{\todo[size=\tiny,color=RoyalBlue!30]{#1 \\ \hfill --- Kel}}

\newcommand{\qbinom}[2]{\left[ \begin{smallmatrix}#1\\#2\end{smallmatrix} \right]}

%remove the comment from the following line to remove all the
% extra proofs:
%\renewcommand{\idiot}[1]{}

\title{Quasisymmetric Varieties and Temperley-Lieb algebras}
\author{Nantel Bergeron,
Lucas Gagnon}
\address{Dept. of Math. and Stat.\\ York  University\\ To\-ron\-to, Ontario M3J 1P3\\ CANADA}
\email{bergeron@yorku.ca}

\thanks{This work is supported in part by York Research Chair and NSERC.
This paper originated in a working session at the Algebraic
Combinatorics Seminar at Fields Institute}
\date{}

\keywords{Quasisymmetric Polynomials, } 

\subjclass[2020]{05E05, ???}

\begin{document}

\begin{abstract}
\end{abstract} 

\maketitle

%%%%%%%%%%%%%%%%%%%%%%%%%%%%%%%%%%%%%%
%%%%%%%%%%%%%%%%%%%%%%%%%%%%%%%%%%%%%%
%%%%%%%%%%%%%%%%%%%%%%%%%%%%%%%%%%%%%%
\section{Quasisymmetric vanishing polynomials}
In this section we define polynomials in $n$ variables that vanishes on the set $QSV_n$ and such that
the homogeneous top degree is quasisymmetric. More precisely, let $\alpha=(\alpha_1,\alpha_2,\ldots,\alpha_k)$
be any fixed composition of $d>0$. Let 
	$$M_\alpha(x_1,x_2,\ldots,x_n)=\sum_{1\le i_1<i_2<\cdots<i_k\le n} x_{i_1}^{\alpha_1} x_{i_2}^{\alpha_2}\cdots  x_{i_k}^{\alpha_k}$$
be the quasisymmetric monomial indexed by $\alpha$. We will define a polynomial $P_\alpha(x_1,x_2,\ldots,x_n)$ such that
  $P_\alpha = M_\alpha +$  lower degree terms, and  $P_\alpha(\sigma)=0$  for all $\sigma\in QSV_n$.

$1=f_1<f_2<\cdots <f_\ell<f_{\ell+1}=n+1$ and 
$\beta_i=\alpha_{f_i}+\alpha_{f_i+1}+\cdots+\alpha_{f_{i+1}-1}$.
When $\alpha$ refine $\beta$, we write $\alpha\le\beta$.

\begin{definition}label{def:vanishP}
 For any $\alpha=(\alpha_1,\alpha_2,\ldots,\alpha_k)$, the polynomial $P_\alpha(x_1,x_2,\ldots,x_n)$ is defined as
	$$P_\alpha=\sum_{\ell=1}^k (-1)^{k-\ell} \sum_{1=f_1<f_2<\cdots <f_{\ell+1}=k+1  \atop 1\le i_1<i_2<\cdots < i_\ell\le n} \prod_{j=1}^\ell 
	      \Big( (x_{i_j}^{\alpha_{f_j}}- i_j^{\alpha_{f_j }})  i_j^{\alpha_{f_j +1}+\cdots+\alpha_{f_{j +1}-1}} \Big)
	      $$
	      The top degree of $P_\alpha$ is in the sum when $\ell=k$, we must have $f_i=i$ in this case. Choosing the variable $x_{i_j}$ in all binomials $(x_{i_j}^{\alpha_{f_j}}- i_j^{\alpha_{f_j }})$, we get $M_\alpha$.
\end{definition}
	      
For a fix $\sigma$, we will partition the sums in $P_\alpha$ according to the cycle type of $\sigma$ and the non-crossing structure
of these cycle will play a major role in showing the following theorem.

\begin{theorem}\label{thm:vanishing}
 For any $\alpha$ and any $\sigma\in QSV_n$ we  have $P_\alpha(\sigma)=0$.
\end{theorem}

\begin{proof}
 Let $\sigma=C_1C_2\cdots C_r$  the decomposition of $\sigma$ into disjoint cycle. We include the fix points a 1-cycles.
 Given a set on indices $1\le i_1<i_2<\cdots < i_\ell\le n$ we say that the cycle support of $S=\{ i_1,i_2,\ldots , i_\ell\}$
 is $C(S)=\{j : S\cap C_j \not= \emptyset \}$. We have
 	$\displaystyle P_\alpha=\sum_{T\subseteq [r]} P_{\alpha,T},$
where 
\begin{equation}\label{eq:PT}
P_{\alpha,T}=
	\sum_{\ell=|T|}^k (-1)^{k-\ell} \hskip-.8cm \sum_{{1=f_1<f_2<\cdots <f_{\ell+1}=k+1  \atop 1\le i_1<i_2<\cdots < i_\ell\le n}\atop C(\{ i_1,i_2,\ldots , i_\ell\})=T} \prod_{j=1}^\ell 
	      \Big( (x_{i_j}^{\alpha_{f_j}}- i_j^{\alpha_{f_j }})  i_j^{\alpha_{f_j +1}+\cdots+\alpha_{f_{j +1}-1}} \Big)\,.
\end{equation}
We show that $P_{\alpha,T}(\sigma)=0$ for all $T$. If $T=\emptyset$ there is nothing to show as $P_{\alpha,\emptyset}=0$. 
We first consider the case when $|T|=1$, and then use the non-crossing structure of the cycles to reduce the case $|T|>1$
to $|T'|=1$.

\medskip
\noindent{\bf Case $\bf |T|=1$}: Let $T=\{t\}$ and $C_t=(a_m\,\ldots\,a_2\,a_1)$, where $a_1<a_2<\cdots<a_m$. Remark that on such cycle, the variable $x_{a_i}=a_{i-1}$ with the convention that $a_0=a_m$.
Expanding all the product in the definition of $P_{\alpha,T}$ and evaluating at $\sigma$, we obtain
\begin{equation}\label{eq:PTatsigma}
	P_{\alpha,\{t\}}(\sigma)=
	\sum_{{1\le j_1\le j_2\le\cdots\le j_k\le m}} \sum_{{\epsilon_1,\epsilon_2,\ldots,\epsilon_k \atop \epsilon_i\in \{0,1\}}\atop \epsilon_i=1 \text{ if } j_{i-1}=j_i} (-1)^{\sum \epsilon_i}\ 
	   z_{j_1,\epsilon_1}^{\alpha_1}z_{j_2,\epsilon_2}^{\alpha_2} \cdots z_{j_k,\epsilon_k}^{\alpha_k}\,,
\end{equation}
where $z_{j_i,\epsilon_i}=a_{j_i-1}$ if $\epsilon_i=0$, otherwise $z_{j_i,\epsilon_i}=a_{j_i}$. To show that $P_{\alpha,\{t\}}(\sigma)$, we construct a sign reversing involution 
on the evaluation of the terms in Equation~\eqref{eq:PTatsigma}.

Given $1\le j_1\le j_2\le\cdots\le j_k\le m$ and $\epsilon_1,\epsilon_2,\ldots,\epsilon_k$, then the term 
$$z_{j_1,\epsilon_1}^{\alpha_1}z_{j_2,\epsilon_2}^{\alpha_2} \cdots z_{j_k,\epsilon_k}^{\alpha_k}=a_{s_1}^{\alpha_1}a_{s_2}^{\alpha_2}\cdots a_{s_k}^{\alpha_k},$$
where $s_i\in\{j_i,j_i-1\}$ depending on $\epsilon_i$ and  $0\le s_1\le s_2\le \cdots\le s_k\le m$. First find the largest, rightmost $s_i\not\in\{0,m\}$. 

\medskip
\noindent
If $\epsilon_i=1$, then let $\epsilon'_i=0$ and $j'_i=j_i+1$, fixing all other values $\epsilon'_p=\epsilon_p$ and $j'_p=j_p$ for $p\not= i$.

\medskip
In this case remark that $z_{j_i,\epsilon_i}=a_{j_i}=z_{j'_i,\epsilon'_i}$ and $z_{j_p,\epsilon_p}=z_{j'_p,\epsilon'_p}$ for  $p\not= i$. Hence
 $$ (-1)^{\sum \epsilon'_i}\ z_{j'_1,\epsilon'_1}^{\alpha_1}z_{j'_2,\epsilon'_2}^{\alpha_2} \cdots z_{j'_k,\epsilon'_k}^{\alpha_k}
  = - (-1)^{\sum \epsilon_i}\ z_{j_1,\epsilon_1}^{\alpha_1}z_{j_2,\epsilon_2}^{\alpha_2} \cdots z_{j_k,\epsilon_k}^{\alpha_k}$$
 To show that $z_{j'_1,\epsilon'_1}^{\alpha_1}z_{j'_2,\epsilon'_2}^{\alpha_2} \cdots z_{j'_k,\epsilon'_k}^{\alpha_k}$ is a term of the sum Equation~\eqref{eq:PTatsigma}
 we need to show that $j'_1\le\cdots\le j'_{i}\le j'_{i+1}\le\cdots\le j'_{k}$ and since $\epsilon'_i=0$, we need $j'_{i-1}<j'_i$.
 For the last inequality, we have $j'_{i-1}=j_{i-1}\le j_i<j_i+1=j'_i$. For the other inequality among the $j$'s, we only need to show that $j'_{i}=j_i+1\le j_{i+1}= j'_{i+1}$.
 Here we recall that $i$ is chosen so that $s_i$ is the rightmost values such that $0<s_i<m$. This implies that either $i=k$ and there is no $j_{i+1}$ or $s_{i+1}=m\in\{j_{i+1},j_{i+1}-1\}$
 and $j_{i+1}\ge m >s_i=j_i$. The last equality follows from $\epsilon_i=1$. We thus have that all such terms cancelled each other in Equation~\eqref{eq:PTatsigma}.
 
 The case where $\epsilon_i=0$ is very similar and is the reverse of the operation above. The choice of the rightmost $s_i\not\in\{0,n\}$ will be the same in both cases, showing that we indeed
 have a sign reversing involution. All terms such that the evaluation $a_{s_1}^{\alpha_1}a_{s_2}^{\alpha_2}\cdots a_{s_k}^{\alpha_k}$ contains some $0<s_i<m$ will cancel.
 The only two terms that survive the cancelation are
 $$(-1)^{n-1}a_{0}^{\alpha_1}a_{m}^{\alpha_2}\cdots a_{m}^{\alpha_k}+  (-1)^n a_{m}^{\alpha_1}a_{m}^{\alpha_2}\cdots a_{m}^{\alpha_k}$$
 that can only be obtained when $j_1=1$ and $j_2=\cdots=j_k=m$ with $\epsilon_1=0$ and $\epsilon_2=\cdots=\epsilon_k=1$, for the first term;
 and when $j'_1=\cdots=j'_k=m$ with $\epsilon'_1=\cdots=\epsilon'_k=1$ for the second term. Since $a_0=a_m$:
  $$P_{\alpha,\{t\}}(\sigma)=(-1)^{n-1}a_{0}^{\alpha_1}a_{m}^{\alpha_2}\cdots a_{m}^{\alpha_k}+  (-1)^n a_{m}^{\alpha_1}a_{m}^{\alpha_2}\cdots a_{m}^{\alpha_k}=0.$$


\medskip 
\noindent{\bf Case $\bf |T|>1$}: 
When we have more than one cycle involved, let $T=\{t_1,t_2,\ldots,t_r\}$ and we assume (without lost of generality) that $C=C_{t_1}$
is a cycle that do not contain (in the non-crossing sense) any nested cycles among $C_{t_j}$ for $j>1$. The fact that the cycles of $\sigma$ are non-crossing guaranties the existence of such $C$
for any given $T$.  We now partition the terms of Equation~\eqref{eq:PT} according to the intersection of $ i_1<i_2<\cdots < i_\ell$ with the $C_{t_j}$ for $j>1$ and the corresponding possible choices of $f_j$'s.
We  show that the portion of the terms intersecting $C$ is a vanishing polynomials as in the case $|T|=1$. Let $c=\min(C)$ and $d=\max(C)$.
Assume we have $C(\{ i_1,i_2,\ldots , i_\ell\})=T$ and let 
 	$$CQ_j(\{i_1,i_2,\ldots,i_\ell\})=\{i_1,i_2,\ldots,i_\ell\}\cap C_{t_j}\not= \emptyset\,.$$ 
From our choice of $C=C_{t_1}$, we have $CQ_j(\{i_1,i_2,\ldots,i_\ell\})=\{i_1,i_2,\ldots,i_\ell\}\cap \{i: c\le i\le d\}$. Outside the range $[c,d]=\{i: c\le i\le d\}$, we fix all the other parameters involved in the terms
$P_{\alpha,T}$ in Equation~\eqref{eq:PT}.
Fix $Q=(Q_2,\ldots,Q_r)$ where $\emptyset\not= Q_j\subset C_{t_j}$ such that 
	$$\bigcup_{j=2}^r Q_j = \{\underline{i}_1,\underline{i}_2,\ldots,\underline{i}_p\} \cup  \{\overline{i}_1,\overline{i}_2,\ldots,\overline{i}_q\}$$
where $p+q<k$ and 
	$$\underline{i}_1<\underline{i}_2<\cdots<\underline{i}_p<c\le d<\overline{i}_1<\overline{i}_2<\cdots<\overline{i}_q\,.$$
We also fix $F=\{\underline{f}_1,\underline{f}_2,\ldots,\underline{f}_p,\underline{f}_{p+1},\overline{f}_1,\overline{f}_2,\ldots,\overline{f}_q,\overline{f}_{q+1}\}$. where
	$$1=\underline{f}_1<\underline{f}_2<\cdots<\underline{f}_p<\underline{f}_{p+1}<\overline{f}_1<\overline{f}_2<\cdots<\overline{f}_q<\overline{f}_{q+1}=k+1\,.$$
For any term of the sum in  Equation~\eqref{eq:PT}, we have a unique corresponding $Q$ and $F$. 
In particular, we have $\displaystyle P_{\alpha,T}=\sum_Q P_{\alpha,T,Q,F}$ where
\begin{align*}\label{eq:PTQ}
P_{\alpha,T,Q,F}= &\prod_{j=1}^p 
	     	\Big( (x_{\underline{i}_j}^{\alpha_{\underline{f}_j}}-\underline{i}_j^{\alpha_{\underline{f}_j }})  \underline{i}_j^{\alpha_{\underline{f}_j +1}+\cdots+\alpha_{\underline{f}_{j +1}-1}} \Big) \times\\
	&\sum_{\ell=1}^{k-p-q} (-1)^{k-p-q-\ell} \hskip-1cm
	\sum_{{{{\underline{f}_{p+1}=f_1<f_2<\cdots <f_{\ell-p-q+1}=\overline{f}_1   \atop c \le i_1<i_2<\cdots < i_{\ell-p-q} \le d}\atop C(\{ i_1,i_2,\ldots , i_{\ell-p-q}\})=\{t_1\}}}}
		\prod_{j=1}^\ell 
	     	\Big( (x_{i_j}^{\alpha_{f_j}}- i_j^{\alpha_{f_j }})  i_j^{\alpha_{f_j +1}+\cdots+\alpha_{f_{j +1}-1}} \Big)\times\\
		&\quad\prod_{j=1}^q 
	     	\Big( (x_{\overline{i}_j}^{\alpha_{\overline{f}_j}}-\overline{i}_j^{\alpha_{\overline{f}_j }})  \overline{i}_j^{\alpha_{\overline{f}_j +1}+\cdots+\alpha_{\overline{f}_{j +1}-1}} \Big) \,.
\end{align*}
When we evaluate $P_{\alpha,T,Q,F}(\sigma)$, then centred term above is $P_{(\alpha_{\underline{f}_{p+1}}, \alpha_{\underline{f}_{p+1}+1} , \ldots, \alpha_{\overline{f}_1} ),\{t_1\}}(C_{t_1})=0$ using the result for $|\{t_1\}|=1$ in first part of the proof. This complete the proof.
\end{proof}




\vskip 1in

\begin{thebibliography}{999}


\bibitem{AB} J. C. Aval, N. Bergeron,
\textit{Catalan paths and quasi-symmetric functions}.
Proc. of the Am. Math. Soc., 2003, 131(4), pp. 1053--1062.
\href{https://doi.org/10.1090/S0002-9939-02-06634-0}{10.1090/S0002-9939-02-06634-0}.

\bibitem{ABB} J. C. Aval, F. Bergeron, N. Bergeron,
\textit{Ideals of quasi-symmetric functions and super-covariant polynomials for $S_n$}.
Adv. in Math., 2004, 181 (2), pp. 353--367.
\href{https://doi.org/10.1016/S0001-8708(03)00068-9}{10.1016/S0001-8708(03)00068-9}.


\bibitem{CLO} D. Cox, J. Little, D. O'Shea,
\textit{Ideals, varieties, and algorithms: an introduction to computational
algebraic geometry and commutative algebra}.
Springer Science \& Business Media; 2013 Mar 9.


\bibitem{L} S. X. Li,
\textit{Ideals and quotients of diagonally quasi-symmetric functions}.
Elec. J. Comb., Vol 24, Issue \#3, P3.3.
\href{https://doi.org/10.37236/6658}{10.37236/6658}.

\bibitem{MacSchub} I. G. Macdonald,
\textit{Notes on Schubert polynomials}.
Publications LACIM, vol. 6, Universit\'e du Qu\'ebec \`a Montr\'eal,  (1991) [ISBN 978-2-89276-086-6].



\bibitem{Stanley} R. Stanley,
\textit{Enumerative combinatorics. {V}ol. 2}.
{Cambridge Studies in Advanced Mathematics}, Vol. {62},
{Cambridge University Press, Cambridge}, (1999).


\end{thebibliography}


\end{document}
